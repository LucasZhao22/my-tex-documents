\documentclass[10pt,lang=cn]{elegantpaper}

\usepackage{threeparttable}
\usepackage{ctex}
\usepackage{titlesec} %自定义多级标题格式的宏包
\usepackage{setspace} %设置单倍行距的宏包
\usepackage{makecell}
%图片并排需要的宏包
\usepackage{graphicx}
\usepackage{float} 
\usepackage{subfigure}
% Start of 'ignore natbib' hack
\let\bibhang\relax
\let\citename\relax
\let\bibfont\relax
\let\Citeauthor\relax
\let\textcite\relax
\makeatletter
\DeclareRobustCommand{\MakeUppercase}[1]{{%
      \def\i{I}\def\j{J}%
      \def\reserved@a##1##2{\let##1##2\reserved@a}%
      \expandafter\reserved@a\@uclclist\reserved@b{\reserved@b\@gobble}%
      \protected@edef\reserved@a{\uppercase{#1}}%
      \reserved@a
   }}
\DeclareRobustCommand{\MakeLowercase}[1]{{%
      \def\reserved@a##1##2{\let##2##1\reserved@a}%
      \expandafter\reserved@a\@uclclist\reserved@b{\reserved@b\@gobble}%
      \protected@edef\reserved@a{\lowercase{#1}}%
      \reserved@a
   }}
\makeatother
\expandafter\let\csname ver@natbib.sty\endcsname\relax
% End of 'ignore natbib' hack
\usepackage[citestyle=authoryear,bibstyle=numeric,sorting=nty]{biblatex}
\addbibresource{wpref.bib}

\setCJKmainfont[ItalicFont=仿宋,BoldFont=黑体]{仿宋}

\graphicspath{{figures/}}
% 设置字号命令
\newcommand{\sanhao}{\fontsize{16pt}{\baselineskip}\selectfont} %三号
\newcommand{\sihao}{\fontsize{14pt}{\baselineskip}\selectfont} %四号
\newcommand{\wuhao}{\fontsize{10.5pt}{\baselineskip}\selectfont} %五号
% 设置引用在右上角
\newcommand{\upcite}[1]{\textsuperscript{\textsuperscript{\cite{#1}}}}
% 设置章节编号格式
\titleformat{\section}[block]{\centering\sihao\kaishu}{\zhnum{section}}{0.5em}{}[] %一级标题用一、二、三等编号,标题占3行,4号楷体,居中
\titleformat{\subsection}[block]{\wuhao\fangsong\hspace{2em}}{(\zhnum{subsection})}{0.1em}{}[] %二级标题用(一)、(二)、(三)等编号,标题占2行,5号仿宋体,左空2格
\titleformat{\subsubsection}[block]{\wuhao\fangsong\hspace{2em}}{\arabic{subsubsection}.}{0.1em}{}[] %三级标题用1.、2.、3.等编号,标题占1行,5号仿宋体,左空2格
\titlespacing*{\subsubsection}{0pt}{2pt}{2pt} %设置三级标题占一行并为单倍行距
% 设置摘要字号大小为五号
\renewcommand{\abstractnamefont}{\wuhao\bfseries}

% 设置目录空格
\renewcommand{\contentsname}{\centering {目 \quad 录}}

\makeatletter
% 默认生成长度为4cm的下划线
\newcommand\dlmu[2][4cm]{\hskip1pt\underline{\hb@xt@ #1{\hss#2\hss}}\hskip3pt}
% 设置标题大小为三号
\patchcmd{\@maketitle}{\LARGE}{\sanhao}{}{}
\makeatother

% 标题信息
\title{人口变化与经济发展:一个文献综述\vspace{-2em}} %减少标题下方的空白
\date{}

\begin{document}

% 封面页
\begin{titlepage}
    \begin{center}
        \begin{figure}[!h]
            \centering
            \includegraphics[width=.7\textwidth]{logo}
        \end{figure}
        
        \vspace*{80pt}

        \begin{tabular}{cc}
            \sihao \kaishu{学年学期:}&
            \sihao {\dlmu[8cm]{2020——2021学年第二学期}}\vspace{12pt}\\
            \sihao \kaishu{课程名称:}&
            \sihao {\dlmu[8cm]{发展经济学}}\vspace{12pt}\\
            \sihao \kaishu{课程代码:}&
            \sihao {\dlmu[8cm]{0510153}}\vspace{12pt}\\
            \sihao \kaishu{任课教师:}&
            \sihao {\dlmu[8cm]{赵文哲 \ 梁银鹤}}\vspace{12pt}\\
            \sihao \kaishu{姓 \qquad   名:}&
            \sihao {\dlmu[8cm]{赵之航}}\vspace{12pt}\\
            \sihao \kaishu{学 \qquad   号:}&
            \sihao {\dlmu[8cm]{2018311178}}\vspace{12pt}\\
            \sihao \kaishu{班  \qquad  级:}&
            \sihao {\dlmu[8cm]{国民经济管理18}}\vspace{40pt}\\

            \sihao \kaishu{总 \qquad   分:}&
            \sihao {\dlmu[8cm]{}}\vspace{12pt}\\
            \sihao \kaishu{评 \hfill 分 \hfill 人:}&
            \sihao {\dlmu[8cm]{}}
        \end{tabular}
        
    \end{center}

    \clearpage

\end{titlepage}

\maketitle

% \tableofcontents
% \newpage
\pagenumbering{arabic} %罗马数字页码

\begin{abstract}
    \linespread{0.91} %设置单倍行距
    \wuhao\fangsong %设置字体和字号
    \indent 人口因素是我国经济发展的重要因素之一,总结人口变化与经济发展的关系意义重大。我们分别对国内外的研究进行总结和分析。首先,我们研究国外对人口增长与经济增长之间关系的成果。
    我们在总结成果的基础上,对当今全球化世界影响下,由于人口因素导致的不平等和人口再分配问题进行探讨。
    接着我们回到中国,总结国内外对中国人口变化的研究,并对中国的计划生育、人口老龄化以及放开生育等政策和问题进行探讨和分析,并给出了一些建议。
\keywords{\fangsong{人口变化\quad 经济发展\quad 人口政策}}
\end{abstract}

\begin{spacing}{1.06} %设置正文单倍行距
    \wuhao %设置字号
\section{引言}

人口问题、不平等与经济发展一直是全世界所面对的重大问题之一,人口变化与经济发展之间的联系也已经被众多经济学家广泛研究过。一些经济学家认为,未来高收入国家的经济增长会放缓,由于这些国家
的生育率下降,人口增长会放缓甚至出现负增长。另一部分人则认为由于地球上的有限资源已经被过度利用,全球的人口增长都会呈现放缓趋势。\parencite{peterson2017role}
人口的变化会对一个国家的年龄结构、劳动力市场、不平等以及移民的开放程度产生直接或间接的影响,而这些因素又会对经济发展产生影响。在当下,经济全球化已经变得不可或缺,尽管出现了一定程度的
逆全球化现象,但全球化的趋势仍然没有改变,一些经济学家认为全球化也会对人口和经济发展以及其他影响本文所讨论的这两大主题的因素产生直接或间接的作用。\parencite{kentor2001long}
另一方面,在我国人口老龄化趋势明显,国家开放三孩政策的今天,研究人口变化与经济发展的关系,对展望我国未来经济发展的前景有着重要意义。本文将结合国内外对人口变化与经济发展的研究,
总结人口变化与经济发展之间的联系,并探讨我国的老龄化问题与放开生育政策对经济增长的作用。

\section{国外研究}

\subsection{人口增长与经济增长之间的关系}

\Citeauthor{peterson2017role}在\citeyear{peterson2017role}年的研究中分析了不同国家人口增长与GDP增长率、人均GDP增长率的统计关系,总结了各种经济模型中对人口的处理,发现不同的模型中
有一个相同点,即人口增长与人均产出增长互相不独立,两者之间的关系在不同情况下大不相同。例如,一个国家的人口结构会对两者的关系产生重要的影响。在日本等老龄化严重的国家,除非生产力得到极大的
发展,否则会出现较少的劳动人口支持较多的退休人口,因此经济增长将会放缓;在非洲等出生率和死亡率都比较高的国家,需要大量的适龄劳动人口支持儿童的教育、医疗等需求,而这些儿童在长大后会补充
劳动力市场,由于高死亡率国家的养老问题并不迫切,因此经济增长预期将会加快。人口增长在世界各国的趋势都不大相同,尽管人口政策会对人口年龄结构产生一定的影响,但这样的影响需要较长的时间
才能反映在劳动力市场上,因此人口增长趋势在一定时间内不会发生剧烈的变化,人口增长对人均经济增长具有高度的国别性。

在发达国家,人口增长放缓,劳动力市场在未来可能会出现萎缩。另一方面,发展中国家通常会有着更高的人口增长,本国的劳动力市场不能完全支持全部劳动力的就业。这样的情况促进了劳动力从发展中国家
到发达国家的流动。对移民更加开放的国家,不论是高收入国家还是低收入国家,都可能收益,从而允许更多的人移民。过往的研究指出,国家与国家之间的不平等远大于国家内部区域之间的不平等,这由于
发达国家的公民享受了“身份溢价”。发展中国家的收入上升以及国际间的移民有助于缓解这一问题,但开放移民的举措会招致发达国家公民的反对,从而使发达国家的移民政策变得更加严格。特朗普政府的政策
便是一个典型的例子。

一部分人认为由于资源的过度利用,人口增长将对环境产生较大的压力,减少人口总数对维持可无限期持续的人口数量至关重要。这些分析的前提是人类社会科技发展的停滞,未来的技术创新不足以解决目前
面对的资源和环境问题。这样的假设不论从历史的角度还是未来的角度都是缺乏依据的。以自然资源为例,石油资源的减少会导致其价格上涨,从而刺激技术创新转向创造石油替代品的领域,进而解决石油枯竭
的问题,电动汽车近年来的发展佐证了这一论断。从政策方面,对化石燃料提高税收也会刺激市场减少对化石燃料的利用,对环境和技术进步都会产生好出。因此,认为人类的技术创新在未来无法解决目前面对的
资源问题是错误的,在可预期的未来内,人口总数不会减少。

\subsection{全球化、不平等以及人口再分配}

随着全球化程度的不断加深,各国之间的联系也变得更加紧密。对于这种紧密的联系,经济学家产生了两种互相对立的观点。现代化理论认为,发展是所有国家必经的过程,发达国家的援助将会加快这一进程,
在这一过程中发达国家对发展中国家产生的负面影响,只是每个国家都会经历的“成长阵痛(growing pains)”的一部分。依赖理论则认为,发展不是一成不变的,当前的政治、经济、军事、文化环境与发达国家
发展的阶段截然不同,发达国家的援助反而会阻碍欠发达国家的发展,使得欠发达国家过于依赖发达国家的资本和商品。劳动密集型产业和资本密集型产业在全球的分布支持了依赖理论的观点。\parencite{kentor2001long}

\citeauthor{kentor2001long}在\citeyear{kentor2001long}年的研究中指出,外国资本的投资可能会导致不平等的加剧。首先,外国资本通过少部分高薪精英来管理投资的产业,扭曲了被投资国家的阶级结构。
其次,这些投资产生的岗位大部分是低技术、低薪水的工作,并不能提高劳动人口的质量。此外,外国资本投资产生的收益大部分回到了投资国,这阻碍了本国资本的形成。最后,外国资本会创造有利于投资国的
政治经济环境,阻碍被投资国的工人获取更高工资和更多福利的机会。上述因素会导致欠发达国家在对全球化依赖加深的同时,本国的劳动力市场却没有得到发展,以拉丁美洲为例,一些拉丁美洲国家在经济发展的
同时,本国人民却没有享受到经济发展的福利。

全球化带来的不平等进一步会反映在人口结构上。由于欠发达国家的人民没有享受到经济发展的福利,劳动力市场仍以劳动密集型产业为主,因此欠发达国家的生育率不会下降。通常,经济发展会使得生育率出现一定程度
的下降,这由于父母从子女身上获得福利的愿望减少。收入不平等对生育率的影响十分积极,甚至达到了收入增长对生育率影响的两倍,而这样的人口增加会加剧本国劳动力市场劳动密集产业集中的情况,从而进一步
提高不平等的程度。不平等的加剧会使得欠发达国家的政治环境动荡,进而影响经济增长。

\begin{figure}[!h]
    \centering
    \includegraphics[width=.6\textwidth]{人口流动}
    \caption{1995-2000年突出的省际人口净流动。}
    \label{fig:migration}
\end{figure}

当然,我们也要注意到全球化对发展中国家带来的好处。以中国为例,我国在改革开放后顺应全球化的趋势,东部沿海地区由于对外贸易的增加经济迅速发展。区域经济的发展反映到人口上便是国内的人口再分配。
如图\ref{fig:migration}所示,经济发展不平衡使得人口从中南部和西南部较为贫困的省份流向最发达的东部沿海省份。人口流动与区域间的经济差异出现了明显的相关性。
1995-2000年,从中西部地区向东部地区的移民流量呈指数级增长。发展水平居全国前列的北京、天津和东南沿海地区是大多数人选择的目的地。
尤其是改革开放以来经济增长显著的广东,成为了吸引力极强的人口流动目的地。唯一的例外出现在西部地区,新疆在90年代的高经济增长率使得其从人口流动中获得了人口。
相比之下,中国最贫困地区中的几个中南部和西南部省份成为最大的人口流出地。集中的迁移模式反映了省际经济发展的异质性增加。\parencite{fan2005interprovincial}

不平等、人口增长和经济发展是相互嵌套,互有影响的,不能单独剥离开来讨论。全球化对这些因素的影响是双面的,一方面外国资本的渗透阻碍了本国人均收入的增长,加剧了不平等;另一方面全球化带来的
贸易机会使得发展中国家的部分地区经济得到了快速发展。对待全球化,要对不同方面的影响分别进行分析,再给出相对应的政策,才能真正促进本国经济的发展。

\subsection{国外对中国的研究}

中国作为世界上人口最多的国家,同时也是发展中国家里经济发展较好的国家,推出过一系列与人口相关的政策,在研究人口变化和经济发展之间的关系这一课题时具有极大的研究价值。
\textcite{chaurasia2021economic}分析了1990年至2018年间中国和印度的经济增长与人口变化,印度同样是人口大国和经济增长较快的发展中国家,是合适的比较对象。
我们采用\citeauthor{chaurasia2021economic}的分析框架。

记$Y$为价格一定的情况下的国民生产总值(GDP),$P$为人口数量,则$Y$是人口数量与价格一定情况下人均GDP的乘积。
$$
Y=P \times \frac{Y}{P}
$$

人均GDP可进一步写为
$$
\frac{Y}{P}=\frac{Y}{L} \times \frac{L}{W} \times \frac{W}{P}
$$

其中,$L$表示劳动人口(实际参与到劳动中的人口),$W$表示适龄劳动人口。综合上述两个式子,我们得到
$$
Y=P \times \frac{Y}{L} \times \frac{L}{W} \times \frac{W}{P}
$$
$$
Y=(P \times \frac{W}{P}) \times (\frac{Y}{L} \times \frac{L}{W})
$$

该式将经济产出分为了人口成分和经济成分两部分。影响人口成分的因素有两个:人口数量($P$)和适龄劳动人口占比($\frac{W}{P}$),它们反映了人口变化。影响经济成分的要素也有两个:劳动生产率($\frac{Y}{L}$)
和参与机会($\frac{L}{W}$),它们反映了社会系统和经济系统的状况。

令$D=\frac{W}{P},I=\frac{Y}{L},E=\frac{L}{W}$,则经济产出的相对增长可表示为
$$
\frac{Y_2}{Y_1}=\frac{P_2}{P_1} \times \frac{D_2}{D_1} \times \frac{I_2}{I_1} \times \frac{E_2}{E_1}
$$
左右两边取对数并用$r_Y,r_P,r_D,r_I,r_E$表示取对数后的值,我们有
$$
r_Y=r_P+r_D+r_I+r_E
$$

类似地,我们也能得到经济产出的绝对增长。
$$
\nabla Y = Y_2 - Y_1 = \frac{Y_2-Y_1}{ln(Y_2)-ln(Y_1)} \times (ln(Y_2)-ln(Y_1))
$$
另一方面
$$
ln(Y_2)-ln(Y_1) = (lnP_2-lnP_1) + (lnD_2-lnD_1) + (lnI_2-lnI_1) + (lnE_2-lnE_1)
$$
因此
\begin{equation}
    \begin{aligned}
    \nabla Y = Y_2 - Y_1 &= \frac{Y_2-Y_1}{ln(Y_2)-ln(Y_1)} \times (lnP_2-lnP_1) + \frac{Y_2-Y_1}{ln(Y_2)-ln(Y_1)} \times (lnD_2-lnD_1) \\
                    &+ \frac{Y_2-Y_1}{ln(Y_2)-ln(Y_1)} \times (lnI_2-lnI_1) + \frac{Y_2-Y_1}{ln(Y_2)-ln(Y_1)} \times (lnE_2-lnE_1) \nonumber
    \end{aligned}
\end{equation}
即$\nabla Y = \partial P + \partial D + \partial I + \partial E$

增长的人口成分($DC$)为 $DC = \partial P + \partial D$,其中$\partial D$通常被称作人口红利。增长的经济成分($EC$)为$EC = \partial I + \partial E$。

\textcite{chaurasia2021economic}对1990年至2018年间中国和印度的经济增长与人口变化用上述分析框架分析后,发现印度约三分之一的经济增长归因于人口因素,因此对提高印度人民的生活质量几乎没有贡献。
而中国的情况却大不相同,90\%以上的经济增长来自经济因素——劳动生产率和劳动适龄人口参与生产活动的机会。这意味着中国的经济增长对人民生活质量的提高相比印度要强得多。
换言之,经济增长的构成不同,即人口因素与经济因素的占比不同,加剧了两国人民生活水平的差距。

中国能够为适龄劳动人口创造充足的机会,意味着中国已经进入了人口变化的高级阶段,当今中国的经济增长几乎独立于人口因素,主要由经济因素驱动。
中国经济发展的关键在于不断提高劳动生产率,并寻找适龄劳动人口参与社会和经济生产体系的新途径。
进入人口变化的高级阶段的另一个标志为中国的生育水平继续远低于更替水平,中国人的平均寿命接近80岁,人口红利转为负值。
因此,未来几年中国老年人口的比例必将迅速增加。
在这种情况下,国家必须探索老年人口参与经济生产的可能性以维持经济增长。\citeauthor{chaurasia2021economic}认为中国如果要扭转近年来经济增长放缓的趋势,对老年人口的生产利用至关重要。
对老年人口的有效利用也可能是改善老年人健康和福利的关键因素。

\textcite{yao2013empirical}则认为中国目前的经济状况存在步入“中等收入陷阱”的可能,且避免“中等收入陷阱”与老年人口参与生产的关系不显著。\citeauthor{yao2013empirical}对未来中国经济的发展
给出了三点建议。首先,放开生育,保持劳动力供应,避免人口结构短期发生重大变化;其次,促进教育,开发人力资源,建立人力资本存量以推动技术进步;最后,积极推进工业化和现代化,充分利用我国尚未
完全利用的人口红利。

\section{国内研究}

\subsection{计划生育政策的影响}

计划生育无疑是我国最重要的国策之一,在短短数十年间改变了我国的人口结构,人口增长模式从“高出生、高死亡 、高增长” 转变到“低出生 、低死亡、低增长”,对我国的经济发展造成了很大的影响。
\textcite{汪伟2010计划生育政策的储蓄与增长效应}对计划生育政策下我国人均收入增长率和储蓄率的变化进行了分析,对我国经济发展中人口增长率、生育率的下降与经济增长率、储蓄率的上升同步
的现象做了一定的解释。分析表明,人口政策很可能是储蓄率和人均收入增长率上升的一个重要原因。计划生育政策实行前,平均每个家庭要养育接近6个孩子,这意味着家庭几乎不可能储蓄,从而导致了
物质积累的不足,每个孩子能够分到的资源有限,进而导致了人力资源的不足。计划生育实施后,生育率的下降自然会使得储蓄率和经济增长率的上升。计划生育政策的实行有力避免了我国陷入低水平的经济
增长,使得我国没有掉入马尔萨斯均衡陷阱。同时,我们要注意的是,生育率下降的空间已经非常有限,计划生育政策对储蓄和经济增长的边际效应会减弱,继续严格执行计划生育政策的理由在当今可能并不
适用。

另一方面,人口老龄化也是计划生育改变我国人口结构的后果之一。目前我国人口老龄化还尚不严重,处于人口老龄化的初期。随着老龄化程度的加深,我国的适龄劳动人口将会减少,劳动力市场萎缩,而
大量的老年人赡养问题将会成为我国的一大负担。因此,人口老龄化会是我国未来面临的一个重大问题,需要从政策方面进行预防。

\subsection{人口老龄化与放开生育政策}

从上一节的分析我们得到,人口老龄化将会是阻碍我国经济发展的一个重大问题。我国政府对此也出台了相关的政策进行应对,即逐步放开生育的“二孩政策”以及今年宣布的“三孩政策”。
那么这些政策在可预见的未来是否能够改善甚至避免人口老龄化带来的经济增长放缓,是我们关心的问题。\textcite{王浩名2018全面二孩政策下人口结构转变对宏观经济的长期影响}的
研究认为,放开生育政策会促进出生人口的回升,从长期来看会促进经济增长,但同时我们也要注意到,出生人口增长乏力和现有劳动力退出劳动力市场等因素会使得中老年人口结构转变不利于
经济增长。

\citeauthor{王浩名2018全面二孩政策下人口结构转变对宏观经济的长期影响}对放开生育政策提出了三点建议。首先,放开生育不代表放弃管控,要避免“婴儿潮”的出现,否则会出现放开与管控循环的
空转现象。其次,当今育儿成本的提高使得放开生育政策的前景无法估计,需要配套的政策降低育儿成本才能使放开生育的政策真正发挥作用。再次,虽然人口因素在我国经济发展中起了至关重要的因素,
但同时我们要注意避免单一因素决定论,不能将经济发展寄托于人口红利之上。

我们对放开生育的前景是较为乐观的,\textcite{王民祥2019二孩政策对中国经济增长的影响研究}的研究表明我国的人口增速仍存在较大的增长空间,我国的经济和税收政策可以向教育、医疗领域
适当倾斜,以降低育儿成本,提高劳动者素质,进而促进经济的发展。

\section{结语}

人口变化与经济发展息息相关,人口的增长、结构变化、区域流动都对经济发展有着至关重要的影响。我国的经济发展中,人口因素起到了重要的促进作用,研究人口变化对经济发展的影响,对总结过去
的经验,展望未来的发展都具有重要的作用。本文通过对过往研究的分析和总结,阐明了人口增长与经济增长之间的关系以及经济发展中的不平等和人口再分配等问题,对国内外的研究分别进行了探讨。
最终,我们通过总结的研究成果,对当下中国的人口变化与经济发展的重要问题给出了一些建议。在这一过程中,我们发现不同的研究尽管给出的建议不尽相同,但对中国经济发展中的人口因素
分析是相一致的,这证明了计划生育政策作为我国基本国策的正确性,同时也说明了放开生育的合理性。另外,我们要注意的是,人口变化与经济发展之间的关系是复杂的、动态的,在分析时要避免
只关注人口因素、忽视其他因素的情况,否则分析的结果会失去参考价值。

\end{spacing}

\printbibliography[title={参考文献}]

\end{document}