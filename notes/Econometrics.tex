%!TEX program = xelatex
\documentclass[cn,blue,pc,14pt]{elegantnote}
\usepackage{ctex}

\setCJKfamilyfont{hwxk}{STXingkai}             %使用STXingkai华文行楷字体
\newcommand{\huawenxingkai}{\CJKfamily{hwxk}}
\setCJKfamilyfont{hwcy}{STCaiyun}             %使用STCaiyun华文彩云字体
\newcommand{\huawencaiyun}{\CJKfamily{hwcy}}
\setCJKfamilyfont{hwhp}{STHupo}             %使用STHupo华文琥珀字体
\newcommand{\huawenhupo}{\CJKfamily{hwhp}}


\title{\textcolor{black}{计量经济学笔记}}

\author{\textcolor{black}{赵之航}}
\institute{\textcolor{black}{中央财经大学}}

\date{\zhtoday}

\begin{document}

\maketitle

\centerline{
    \includegraphics[width=0.2\textwidth]{Cufenew.jpg}
}


\section{计量经济学定义与stata命令入门}

\textbf{计量经济学}是一门基于统计方法的发展来估计经济关系、检验经济理论、评价和实施政府和商业政策的一门学科。

\subsection{经济数据结构}

\begin{itemize}
    \item \textbf{横截面数据}:在给定时点对个人、家庭、企业、城市、州、国家或一系列其他单位采集样本所构成的数据渠。(随机抽样)
    \item \textbf{时间序列数据}:由对一个或几个变量不同时间的观测值所构成。(数据频率)
    \item \textbf{混合横截面数据}:有些数据既有横截面数据的特点。又有时间序列的特点。例如,假设对美国的家庭进行了两次横截面数据的调查,一次在1985 年,一次在1990 年。
    在1985年,对家庭的一个随机样本调查了工资、储蓄、家庭规模等变量。到了1990 年.用同样的调查问题又对家庭的一个新随机祥本进行调查。为了扩大我们的样本容量,可
    以将这两年的数据合并成一个混合横截面数据。
    \item \textbf{面板数据}:由数据集中每个横截面单位的一个时间序列组成。
\end{itemize}

\subsection{stata常用命令}

\begin{lstlisting}[frame=none]
    describe(d)
    list(l) 变量名,变量名,... (in 起始数字/结束数字) (if 条件)
    drop if
    keep if
    count if
    summarize(su) 变量名,变量名,... (if 条件)
    su 变量名,detail
    mean 变量名,变量名,...
    pwcorr a b c,sig star(.05) {选择项“sig”表示显示相关系数的显著性水平(即p值,列在相关系数的下方),选择项“star(.05)”表示给所有显著性水平小于或等于5%的相关系数打上星号。}
    scatter(sc) tc q {散点图}
    tabulate(ta) pl {经验累积分布函数}
    generate(g,gen) n=_n {生成新变量}
    twoway (scatter tc q)(lfit tc q) {线性回归}
    twoway (scatter tc q)(qfit tc q) {二次回归}
    display(di) {计算器}
    regress(reg) lntc lnq lnpl lnpk lnpf {OLS}
    *help(h)
\end{lstlisting}

\begin{note}
    *表示重要,\{\}内的内容为注释
\end{note}

\end{document}