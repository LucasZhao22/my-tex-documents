%!TEX program = xelatex
\documentclass[cn,blue,pc,14pt]{elegantnote}
\usepackage{ctex}


\setCJKfamilyfont{hwxk}{STXingkai}             %使用STXingkai华文行楷字体
\newcommand{\huawenxingkai}{\CJKfamily{hwxk}}
\setCJKfamilyfont{hwcy}{STCaiyun}             %使用STCaiyun华文彩云字体
\newcommand{\huawencaiyun}{\CJKfamily{hwcy}}
\setCJKfamilyfont{hwhp}{STHupo}             %使用STHupo华文琥珀字体
\newcommand{\huawenhupo}{\CJKfamily{hwhp}}
\setlength{\parindent}{4.1em}


\title{\textcolor{black}{实变函数笔记}}

\author{\textcolor{black}{赵之航}}
\institute{\textcolor{black}{中央财经大学}}

\date{\zhtoday}

\begin{document}

\maketitle

\centerline{
    \includegraphics[width=0.2\textwidth]{Cufenew.jpg}
}


\section{集合及其基数}

\subsection{集合及其运算}

\begin{definition}
    一般说来,如果$p(x)$是一个与$x$有关的条件(或命题),则所有合乎这个条件(或使这个命题成立)的$x$所构成的集合便记为$\{x;p(x)\}$。
    又如果$E$是一个事先给定了的集合,则$E[x;p(x)]$便表示$E$中所有使条件$p(x)$满足的$x$所构成的集合,也就是$\{x;x\in E,p(x)\}$
\end{definition}

\begin{theorem}
    $A=B$的充要条件是$A\subset B$,且$B\subset A$
\end{theorem}

\begin{theorem}
    如果$A\subset B,B\subset A$,则$A\subset C$
\end{theorem}

\begin{definition}
    一般说来,如果$\Lambda$是一集合,对于每一个$\lambda \in \Lambda$,都相应地给定了一个集合$A_\lambda$(以$\Lambda$为下标集的)一族集合。
    这时这族集合的交定义为$$\{x;\mbox{对每一个}\lambda \in \Lambda \mbox{,都有}x\subset A_\lambda\}$$
    记为$\mathop{\cap}\limits_{\lambda \in \Lambda} A_\lambda$.如果$\Lambda=\{1,2,\dots,n\} \mbox{或} \{1,2,\dots,n,\dots\}$,
    则上述交就分别简记为$\mathop{\cap}\limits_{i=1}^n A_i$和$\mathop{\cap}\limits_{i=1}^{\infty} A_i$
\end{definition}

\begin{note}
    集族就是集合的集合(以集合为元素的集合)
\end{note}

\begin{theorem}
    集合的运算满足交换律,结合律,分配律以及幂等律。
\end{theorem}

\begin{theorem}
    (1)$A\cap B \subset A \subset A\cup B$\\
    \indent(2)若
\end{theorem}

\end{document}