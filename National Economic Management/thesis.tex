\documentclass[10pt,lang=cn]{elegantpaper}

\usepackage{threeparttable}
\usepackage{ctex}
\usepackage{titlesec} %自定义多级标题格式的宏包
\usepackage{setspace} %设置单倍行距的宏包
\usepackage{makecell}
%图片并排需要的宏包
\usepackage{graphicx}
\usepackage{float} 
\usepackage{subfigure}
% Start of 'ignore natbib' hack
\let\bibhang\relax
\let\citename\relax
\let\bibfont\relax
\let\Citeauthor\relax
\let\textcite\relax
\makeatletter
\DeclareRobustCommand{\MakeUppercase}[1]{{%
      \def\i{I}\def\j{J}%
      \def\reserved@a##1##2{\let##1##2\reserved@a}%
      \expandafter\reserved@a\@uclclist\reserved@b{\reserved@b\@gobble}%
      \protected@edef\reserved@a{\uppercase{#1}}%
      \reserved@a
   }}
\DeclareRobustCommand{\MakeLowercase}[1]{{%
      \def\reserved@a##1##2{\let##2##1\reserved@a}%
      \expandafter\reserved@a\@uclclist\reserved@b{\reserved@b\@gobble}%
      \protected@edef\reserved@a{\lowercase{#1}}%
      \reserved@a
   }}
\makeatother
\expandafter\let\csname ver@natbib.sty\endcsname\relax
% End of 'ignore natbib' hack
\usepackage[citestyle=authoryear,bibstyle=numeric,sorting=nty]{biblatex}
\addbibresource{wpref.bib}

\setCJKmainfont[ItalicFont=仿宋,BoldFont=黑体]{仿宋}

\graphicspath{{figures/}}
% 设置字号命令
\newcommand{\sanhao}{\fontsize{16pt}{\baselineskip}\selectfont} %三号
\newcommand{\sihao}{\fontsize{14pt}{\baselineskip}\selectfont} %四号
\newcommand{\wuhao}{\fontsize{10.5pt}{\baselineskip}\selectfont} %五号
% 设置引用在右上角
\newcommand{\upcite}[1]{\textsuperscript{\textsuperscript{\cite{#1}}}}
% 设置章节编号格式
\titleformat{\section}[block]{\centering\sihao\kaishu}{\zhnum{section}}{0.5em}{}[] %一级标题用一、二、三等编号,标题占3行,4号楷体,居中
\titleformat{\subsection}[block]{\wuhao\fangsong\hspace{2em}}{(\zhnum{subsection})}{0.1em}{}[] %二级标题用(一)、(二)、(三)等编号,标题占2行,5号仿宋体,左空2格
\titleformat{\subsubsection}[block]{\wuhao\fangsong\hspace{2em}}{\arabic{subsubsection}.}{0.1em}{}[] %三级标题用1.、2.、3.等编号,标题占1行,5号仿宋体,左空2格
\titlespacing*{\subsubsection}{0pt}{2pt}{2pt} %设置三级标题占一行并为单倍行距
% 设置摘要字号大小为五号
\renewcommand{\abstractnamefont}{\wuhao\bfseries}

% 设置目录空格
\renewcommand{\contentsname}{\centering {目 \quad 录}}

\makeatletter
% 默认生成长度为4cm的下划线
\newcommand\dlmu[2][4cm]{\hskip1pt\underline{\hb@xt@ #1{\hss#2\hss}}\hskip3pt}
% 设置标题大小为三号
\patchcmd{\@maketitle}{\LARGE}{\sanhao}{}{}
\makeatother

% 标题信息
\title{新冠疫情对劳动力市场的冲击——失业与应对政策的研究\vspace{-2em}} % 减少标题下方的空白
\date{} % 空日期

\begin{document}

% 封面页
\begin{titlepage}
    \begin{center}
        \begin{figure}[!h]
            \centering
            \includegraphics[width=.7\textwidth]{logo}
        \end{figure}
        
        \vspace*{80pt}

        \begin{tabular}{cc}
            \sihao \kaishu{学年学期:}&
            \sihao {\dlmu[8cm]{2020——2021学年第二学期}}\vspace{12pt}\\
            \sihao \kaishu{课程名称:}&
            \sihao {\dlmu[8cm]{国民经济管理}}\vspace{12pt}\\
            \sihao \kaishu{课程代码:}&
            \sihao {\dlmu[8cm]{0511195}}\vspace{12pt}\\
            \sihao \kaishu{任课教师:}&
            \sihao {\dlmu[8cm]{\normalsize 赵丽芬\ 黄乃静\ 赵扶扬\ 严成樑\ 田子方\ 徐翔\ 赵文哲}}\vspace{12pt}\\
            \sihao \kaishu{姓 \qquad   名:}&
            \sihao {\dlmu[8cm]{赵之航}}\vspace{12pt}\\
            \sihao \kaishu{学 \qquad   号:}&
            \sihao {\dlmu[8cm]{2018311178}}\vspace{12pt}\\
            \sihao \kaishu{班  \qquad  级:}&
            \sihao {\dlmu[8cm]{国民经济管理18}}\vspace{40pt}\\

            \sihao \kaishu{总 \qquad   分:}&
            \sihao {\dlmu[8cm]{}}\vspace{12pt}\\
            \sihao \kaishu{评 \hfill 分 \hfill 人:}&
            \sihao {\dlmu[8cm]{}}
        \end{tabular}
        
    \end{center}

    \clearpage

\end{titlepage}

\maketitle

% \tableofcontents
% \newpage
\pagenumbering{arabic} %罗马数字页码

\begin{abstract}
    \linespread{0.91} %设置单倍行距
    \wuhao\fangsong %设置字体和字号
    \indent 我国劳动力市场在疫情短期冲击下受到的影响是广泛而深远的,研究我国在疫情封锁前后失业的变化、以及失业人群的分布与不同类型人群之间的异质性,
    对于解决失业这一重要民生问题以及推进“六稳”之首的“稳就业”具有重要意义,对政策的制定和执行也具有一定的参考价值。
    本文还将国内应对疫情和失业问题的政策与国外的部分主要国家进行了对比,阐述了不同政策的优势与不足,以及我国政策的合理性及其在我国国情下的优越性。

\keywords{\fangsong{新冠肺炎疫情\quad 失业\quad 社会保障}}
\end{abstract}

\begin{spacing}{1.06} %设置正文单倍行距
    \wuhao %设置字号
\section{引言}
    2020年初新冠肺炎疫情的爆发不仅在全球范围内掀起了近十年来最严重的公共卫生危机,也对我国以及全球各个国家和地区的经济都造成了严重的冲击。直到2021年上半年,疫情在全球范围仍未完全缓解,
    300多万人死亡,1.5亿人被感染,病毒甚至进化出了传染性更强的德尔塔毒株,尽管疫苗接种已经在全球范围内紧张进行,疫情的威胁在可预见的未来消失的可能性依旧不大。
    面对疫情,我国采取了全球范围内最为严格的防控措施,不仅迅速控制了疫情,经济也基本恢复,能够支援其他国家抗击疫情。
    相反,欧美国家的防控措施则不够严厉,疫情始终未得到有效控制。
    防控政策的不同反映了不同国家之间制定政策时不同的考量。欧美国家为了保障经济,采取了宽松的防疫措施,而结果却与初衷背道而驰。我国看似会影响经济的政策却在保障经济运行方面卓有成效。
    国家经济的状况对劳动力市场也有着直接的影响,经济萧条时期居民会更有可能失业。
    结合疫情的实际情况,对部分行业和特定人群,失业情形更为严峻。

    在这一特殊时期,保障就业是极为重要的民生问题,也是保持经济复苏平稳运行所必须解决的问题。研究疫情期间的失业问题对恢复正常生产生活,以及日后预防类似问题再次出现有着重要意义。
    本文将结合我国实际情况,探究疫情对劳动力市场冲击所带来的失业问题,并针对不同行业和不同人群进行分析,重点关注受影响较大的群体,找寻其中不平等的原因。同时,我们针对政府实施的
    政策进行分析,关注其效果,厘清哪些防控措施影响了失业,哪些救助政策缓解了失业,以及这些政策的作用机理。另外,我们将与同时期的国外政策相对比,探究解决同一问题时哪种政策更优及其原因。

\section{背景和数据}

\subsection{国内疫情背景}

2019年12月下旬,武汉报告了第一例新冠病毒肺炎病例,武汉市于2020年1月23日关闭离汉通道。几乎在同时,湖北省全省也实施了相同的政策:公共汽车停运,离开省内的火车与航班取消。
在接下来的两至三天内,几乎全国所有其他省份均实施了类似的封锁措施。在城市,政策实行和生活保障依赖于在当地政府指导下由物业管理公司和居民委员会等以小区为单位组成的一线机构,
确保居民在封锁期间待在家里。
2021年4月8日,武汉解封,离汉通道重新开放,机场、火车站、公共交通恢复运行。
然而,在省、市和社区层面,仍然需要严格的防控措施,包括要求“健康码”、体温检查和返工两周隔离。
事实上,在一些省市区,公共交通在几周内仍未完全恢复运行。不少地方实行“封闭式管理”,严格控制小区的出入。
除了在疫情情况严峻的城市进行封锁外,我国政府对人员和资源不足的农村地区也实施了居家隔离的政策。
许多村庄采取相当原始、过于严格、通常不被官方认可的简单粗暴的封锁措施,例如关闭或封锁通往村庄的所有道路来阻止人员进出。
在农村,因担心传播病毒而不被允许返乡不得不返回工作城市租用的公寓的事件也很普遍。
武汉在4月初之前完全与外界隔绝74天。据2月中旬的一项估计,约有1.5亿人受到严格的出行限制,超过7亿人的行动受到一定限制,
许多封锁措施一直持续到5月初。
4、5、6月,全国各地每日确诊病例几乎为零,只有东北部分城市4、5月出现了零星暴发。
6月中旬在北京发生了更大规模的疫情,截至6月下旬仍在继续。北京立即推出了“战时”措施来遏制传播,大幅增加首都的交通限制。
在我国疫情防控取得重大成果的背景下,严格的隔离措施可以认为是值得全球共享的经验。
然而,各国在政策实施的偏向以及政策执行等方面存在显著差异。

\subsection{国内应对政策}

为应对疫情冲击造成的经济下行导致的失业增加,国务院办公厅印发了《关于应对新冠肺炎疫情影响强化稳就业举措的实施意见》,在就业、减税降费、复工复产和困难救济等多个方面提出了具体措施。

可以看出,我国的“稳就业”政策偏向于从供给侧解决问题,通过对小微企业施行优惠政策鼓励其吸收就业,避免企业裁员甚至倒闭,以此降低失业人数。
而针对个体补助方面,数据显示,仅有13\%的失业者得到了失业救济。在失业期间,失业者的主要收入来源为家人(47.5\%)和储蓄(38.1\%),仅有7.6\%靠社保;
从社会救助方面来看,88.2\%没有得到任何形式的社会救助,失业保险仅覆盖了7.7\%,只有1.4\%得到最低生活保障救助,1.2\%申请到了小额贷款,0.2\%接受了职业培训。
从数据中我们可以看出,我国的应对政策对个体的直接救助存在不足。

\subsection{国外疫情背景}

在国内的新冠肺炎疫情基本得到控制的同时,国外的疫情情况却不容乐观。Worldometers世界实时统计数据显示,截至2021年6月15日,
全球累计确诊新冠肺炎人数超过17700万,累计死亡人数约为382.7万人。
新增死亡为6256例。康复为161192250例,重症和危重症病人达到84695例;全球新冠确诊病例超过100万例的国家达28个,94个国家病例超10万例。
美国新冠肺炎累计确诊33880552人,是全球新冠肺炎确诊人数最多的国家,印度位居全球第二。

不同国家针对疫情采取的防控措施不尽相同,但大体方向一致。国内限制人群聚集活动,国外限制出入境,实行旅行限制。
在实施方面侧重点不同,效果也不完全相同。国家具体措施如表\ref{table:fkzc}所示。\parencite{kpmg}
\begin{table}[!htbp]
    \centering
    \begin{threeparttable}
        \caption{各主要国家疫情防控政策或举措}\label{table:fkzc}
        \begin{tabular}{cc}
            \toprule[1.5pt]
            \makebox[0.2\textwidth][c]{国家}	&  \makebox[0.4\textwidth][c]{主要疫情防控政策} \\
            \midrule[1pt]
            美国	 & \makecell[l]{美国将旅行限制范围扩大至英国和爱尔兰、来自这两个国家的\\美国居民将被隔离在指定的机场。
                                    此前,美国宣布暂停除英国\\外所有欧洲国家公民前往美国的旅行,这一措施为期30天。} \\ \midrule[0.5pt]
            法国	 & \makecell[l]{法国所有饭店,电影院,咖啡厅,以及非必需的商场全部关闭。\\只有出售食品的超市和集市,以及药房继续开放。} \\ \midrule[0.5pt]
            西班牙   & \makecell[l]{西班牙将在一定程度上实施封锁,除了药房、食品和其他基本\\必需品之外的商铺将全部关闭,武装力量准备就绪,以抗击新\\冠肺炎疫情。}\\ \midrule[0.5pt]
            韩国     & \makecell[l]{韩国政府从3月15日零时起对来自法国、德国、西班牙、英国\\和荷兰5个欧洲国家的旅客适用特别入境检疫程序。韩国济州\\国际机场国际航班14日全线停飞。} \\ \midrule[0.5pt]
            德国     & \makecell[l]{德国全境16个联邦州已陆续全部决定各级学校停课。其中,萨\\克森州、勃兰登堡州、黑森州表示,下周起虽然将暂停“受教育\\义务”,
                                    但学校依然会开放,不来上课的学生不会被追究责任,\\愿意上课的学生仍可继续。}\\ \midrule[0.5pt]
            日本     & \makecell[l]{日本众议院表决通过了《新型流感等对策特别措施法》修正案,\\并于3月13日经参议院表决通过后正式生效。}\\
            \bottomrule[1.5pt]
        \end{tabular}
        \begin{tablenotes}
            \footnotesize
            \item[*] 截至2020年3月19日
        \end{tablenotes}
    \end{threeparttable}
\end{table}

\subsection{国外应对政策}

面对疫情造成的劳动力市场的冲击,欧美虽然也采取了一定的封锁隔离措施,但整体仍偏向于先保障经济,重视个体救助而非严格防控,与我国的政策导向相反。
例如,英国出台了“新冠长假计划(CoronaVirus Job Retention Scheme)”,即员工居家隔离期间由政府保障其薪水的发放,避免公司为降低损失对公司进行裁员,从而保障就业。
英国还降低了申请失业救济的门槛,使得更多人因此受惠。根据“英国家户长期追踪数据库”的数据,截至2020年4月,逾两百万人受益于该项政策,覆盖了英国接近20\%的从业人员。

美国也颁布了类似的政策,包括向小微企业提供3500亿美元支持,疫情封锁期间保留医保,因封锁休无薪长假可向政府申请失业补助,临时性增加2万亿美金失业保障基金等。
另外,美国政府还出台了“美国拯救计划法案”,其中包括增加最低工资、对部分家庭进行直接转移支付等对个体的直接救助政策。\parencite{中美比较}

\subsection{数据来源\label{sec:data}}

根据CEIC数据库的数据显示(图\ref{fig:unemployment}),我国疫情期间失业率整体走势为先升后降,但截至2021年3月,我国的失业率仍未恢复至疫情前的水平,总体情况并不乐观。
依据人社部公布的数据我们可以大致估算出2020年全国失业人口约为9200万。在这些失业人口中,失业时间超过半年的约占四分之一,平均失业时长高达4.5个月。
此外,2020年3月的失业率数据出现了小幅下降,这意味着我国失业率数据的失灵。这是由于我国使用的是登记失业率数据,需要失业人口自主登记,且登记需要一定的门槛,从登记到反映到数据上需要
一定的时间,而疫情期间的居家隔离导致部分失业人群并未及时前去登记,造成了数据的失灵。
\begin{figure}[!h]
    \centering
    \includegraphics[width=.8\textwidth]{失业率}
    \caption{2020年1月31日——2021年3月1日国内失业率折线图}
    \label{fig:unemployment}
\end{figure}

因此,我们需要能够更好地反映劳动力市场变化的数据集。我们利用人均占有量极高的手机来进行数据调查。\parencite{barwick2020covid}我们挑选人口和GDP较高的城市,这样的城市由于人口密度较高,
容易受到疫情的冲击,且人口的构成类型较为丰富,手机的保有量也较大,是适合我们研究的样本。在本文中,我们选择广东作为我们的样本,广东省包括的主要城市有深圳、广州等,均是中国最发达的城市
之一,且广东在城市重新开放后疫情控制相对较好(2020年),能够更好地体现出疫情对劳动力市场造成的冲击。

我们的数据包括2018年1月至2020年9月广东省7100万用户的通话记录信息以及用户的年龄、性别等人口统计信息。我们利用数据中的两个特性来判断用户是否失业。第一个特性是用户拨打失业救济热线12333
的通话次数和时长,由于一个人可能会多次拨打12333,我们使用拨打12333的人数而非12333接到的呼叫次数。第二个特性是手机的位置信息,我们观察其在工作日中一天的位置变化来判断他的通勤模式,
依据通勤模式在疫情前后发生的变化来判断用户是否失业。值得注意的是,并非所有失业人员都会拨打12333.远程办公也会造成通勤模式的变化,因此两个特征给出的是失业人数的下限和上限,并非准确的
失业人数。不过,我们关心的是疫情对劳动力市场的短期冲击,是一段时间内的变化动态,这是官方数据所无法提供的(由于数据收集的时间间隔较长),因此失业人数在合理范围内上下浮动是可以接受的。
另外,手机号码的注册地与是否拥有本地户口并无关联,外来务工人员也可以注册本地手机号码,导致外来务工人口和本地居民之间的失业差距可能会比数据展现出的更大。

\section{分析}

我们使用双重差分法(DID)对数据进行分析,选取实验组为2020年疫情封锁前后的时间段,对照组为2019年相同的时间段。考虑到该时间段内有春节假期,人口流动较大,预期对实验结果会造成影响,
我们使用农历时间而非公历时间。我们采用\textcite{barwick2020covid}的实证模型。记$c$为信号塔覆盖区域,$i$为实验组(2020年)或对照组(2019年),$t$为事件发生点($t=0$表示疫情封锁日),
事件时期为封锁前60天以及封锁后252天(依据广东省封锁时间线划分)。将事件发生期间以每十天为间隔划分,按照时间先后的顺序标记,记为$q$。疫情冲击下劳动力市场的动态变化模型如下:
$$
y_{cit}=\sum_{q=-5}^{24} \beta_q d_i p_q + \alpha_c + \gamma_i + \eta_t + \xi_{it} + \varepsilon_{cit}
$$
其中,$\beta_q$为事件系数,描述封锁对失业造成的影响;$d_i$是虚拟变量,实验组为1,对照组为0;$p_q$为指示变量,当$t \in [10q+1,10(q+1)]$时为1,否则为0;
$\alpha_c,\gamma_i,\eta_t,\xi_{it}$分别为信号塔范围、实验组/对照组、事件日和节假日的固定效应。值得注意的是,由于使用的是农历时间,节假日的日期在
对照组和实验组中会有所不同。

我们将模型按照\nameref{sec:data}中的两个特性分别进行回归,并对回归结果作分析。

\subsection{依据失业热线的数据进行分析}

对于回归后得到的结果,我们重点关注$\beta_q$的估计值,它描述了2020年拨打失业热线的人数相对于2019年的变化百分比。回归结果如图\ref{fig:reg call}所示。从图中我们可以看出,
在疫情封锁前事件系数的估计值在0附近上下波动,在统计学意义上不显著,这表示在疫情封锁前,劳动力市场并未受到强烈的冲击。失业人数在疫情封锁期间有所下降,但在重新开放后迅速增加,
拨打失业热线的人数从3月中旬到9月增加了20\%到50\% ,这与事实也相契合。
\begin{figure}[!h]
    \centering
    \includegraphics[width=.6\textwidth]{回归1.1}
    \caption{失业热线回归结果}
    \label{fig:reg call}
\end{figure}

具体数据显示,疫情封锁期间拨打失业热线的人数减少了37\%,在重新开放后的第一个阶段增加了25\%,第二个阶段增加了45\%。
通话时长的增减趋势也相类似,时长的增加可能由于外来务工人员需要更多的信息导致。总体来说,疫情对劳动力市场的冲击导致拨打失业热线的人数增加了27\%,又因为我们的变量控制以及
固定效应是有限的,根据失业热线得到的分析结果是真实失业人数的下限,但这不影响数据的有效性。

\subsection{依据通勤模式变化的数据进行分析}

对于通勤模式变化的数据,我们认为在一定时间内至少去过一次办公场所的人员不是失业人员,这段时间可以是一周、两周、一个月。由于疫情期间部分岗位采取了居家办公的做法,可能存在一些未
失业人员没有通勤记录,但考虑到解除封锁后,能够完全依靠远程办公的公司数量较少,且我们给出的时间窗口相对较长,我们可以认为居家办公对于以通勤模式判断是否失业的方法影响可忽略不计。

不失一般性,我们选取两周作为时间窗口。通过类似上一节的分析可得,通勤人数在疫情封锁、初步解封和完全解封期间分别减少了29\%,8\%,4\%。2020年相比2019年,没有通勤的人数整体增加了
63\%,即失业的人数增加了63\%。但是,考虑到部分企业从居家办公到实地办公的转换,以及恢复经营需要时间,该数据可能高估了失业人数的上涨,因此我们认为该数据为疫情对劳动力市场冲击造成
失业的影响上限。

另外,我们注意到时间窗口的选取对分析结果几乎没有影响,这应证了我们认为远程办公不会影响数据可靠性的假设。依据通勤模式变化的数据进行分析相比依据失业热线的数据进行分析更可靠,这由于
失业热线的数据更类似“登记失业率”,而通勤模式变化的数据更类似“调查失业率”,前者由于存在门槛(手续繁杂,缺乏信息,心理因素等)不可避免会对数据造成偏差,而后者则不存在这样的问题。\parencite{登记失业率下降}

\subsection{异质性分析}

为了检验疫情对劳动力市场的冲击对不同类型人群的影响,我们对男性和女性、40岁以上和40岁以下、外来务工人群和本地居民三个不同类型人群分别做与上述过程相同的分析,得到的结果
如图\ref{Fig.main}所示。由图\ref{Fig.sub.1}可知,女性比男性受到的影响更大,可能的原因是由于疫情期间学校、托儿所等机构关闭,家庭需要女性照看孩子。

\begin{figure}[H]
    \centering  %图片全局居中
        \subfigure[男性和女性失业情况对比]{
        \label{Fig.sub.1}
        \includegraphics[width=0.45\textwidth]{回归1.2}}
        \subfigure[40岁以上和40岁以下失业情况对比]{
        \label{Fig.sub.2}
        \includegraphics[width=0.45\textwidth]{回归1.3}}
        \subfigure[外来务工人群和本地居民失业情况对比]{
        \label{Fig.sub.3}
        \includegraphics[width=0.45\textwidth]{回归1.4}}
    \caption{异质性分析回归结果}
    \label{Fig.main}
\end{figure}

由图\ref{Fig.sub.2}可知,大龄从业者相比较年轻的从业者受到疫情影响更严重,40岁以上拨打失业热线要多出20\%。同样,外来务工人员相比本地居民拨打失业热线的人数高出120\%到220\%,
且图\ref{Fig.sub.3}的趋势表明该趋势没有减弱的迹象。分析结果表明社会弱势群体,尤其是外来务工人员,受到疫情的冲击更大,而这些社会弱势群体所工作的领域也更多地集中于易受疫情
影响的行业,例如旅游业、餐饮业、建筑业等。

从图中我们还可以看出失业热线在解封后总体呈上涨趋势,这是否只是国家应对失业的政策颁布使得民众对失业救济的意识提高?政策的推广自然会使得民众对登记失业救济的意识增强,但若只是政策
导致的上升,不同类型人群之间不会表现出强烈的异质性。因此,我们可以肯定,社会弱势群体对疫情影响下劳动力市场的冲击更敏感。

基于通勤模式变化的数据进行异质性分析的过程类似,得出的结果相同,不再赘述。

\section{结论}

我们的分析表明,疫情封锁对劳动力市场的冲击导致了失业率的上升,且失业率在不同类型的人群中表现出较强的异质性。社会弱势群体,尤其是外来务工人员,对疫情的冲击更为敏感,失业人数也相对
更多。\textcite{che2020unequal}的研究表明,外来务工人员的实际失业人数可能会比我们预想的更多,达到7000到8000万,尽管国内疫情控制在全球处于领先水平,
我国的经济恢复也是最好的,但由于国外疫情防控成果不佳,我国的外贸市场短时间内仍会处于衰退趋势,这将导致大量的外来务工人员在重新开放后在相当一段时间内仍处于失业的状态。
2021年失业率仍未恢复到疫情前的水平证实了模型的判断。

同时,我们注意到失业救济对弱势群体的覆盖不足,\citeauthor{che2020unequal}指出获得失业救济的230万人中,仅有6.7万属于外来务工人员,这说明我国的应对政策在执行层面与需要救济的人群
存在相当程度的不匹配。尽管新推行的政策对于失业救济的力度加大、门槛放宽,真正惠及到的外来务工人员仍是少数,这导致疫情后居民之间收入不平等的加剧,相当一部分家庭将面临可能陷入贫困的
危机。相比之下,欧美国家的应对政策救济的人群涵盖更广泛,对个体的救助力度更大,不过值得警惕的是这样的做法可能会对未来的经济造成一定的风险,受到救助的人群也可能并不亟需帮助(许多家庭
将救助金拿去买非生活必需品、投资股票等)。我国与欧美国家相比,产业分布不同,以美国为例,其产业以服务业为主,创造了美国79\%的就业,中国的服务业占比则不到50\%,而服务业是受疫情冲击
最严重的行业之一。因此,欧美国家的救助政策在我国可能并不适用,且我国疫情控制速度快、效果好,大规模、大范围的救助政策是不必要的。

\section{结语}

本文对新冠疫情冲击下的劳动力市场进行了分析,得到了封锁防控对失业的具体影响以及其中的不平等因素,并阐述了我国的应对政策,将其与欧美国家的政策做了比较。在应对政策方面,
我国政府将“稳就业”排在“六稳”之首,出台了大量措施稳定就业,但从我们的分析结果可以看出,政策与需要救助的人群之间存在一定程度的不匹配,“稳就业”的措施不能精准“锚定”救助群体。
具体来说,就是社会弱势群体、灵活就业人员、新型就业人员中存在“漏出群体”。\parencite{王震2020新冠肺炎疫情冲击下的就业保护与社会保障}
为了保证政策有效实施,政府需要在政策的执行上有一定的侧重,更彻底的解决办法与户籍制度改革相关,但短期来看,在执行政策时发动基层以及民间组织、社会组织对特定“漏出群体”的额外关注,
发挥基层工作人员的主观能动性是有效的。

由于\textcite{barwick2020covid}中的数据存在一定的误差,且对失业人员的估算范围相当大,因此本文对于实际失业情况的估计并不准确,但分析数据后得出的动态趋势表现出的一致性说明
文章的结论仍具有参考价值。对于失业率的统计,我们则仍需要更精确的方法。

\end{spacing}

\printbibliography[title={参考文献}]

\end{document}