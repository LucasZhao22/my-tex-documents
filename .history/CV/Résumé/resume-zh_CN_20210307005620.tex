% !TEX TS-program = xelatex
% !TEX encoding = UTF-8 Unicode
% !Mode:: "TeX:UTF-8"

\documentclass{resume}
\usepackage{zh_CN-Adobefonts_external} % Simplified Chinese Support using external fonts (./fonts/zh_CN-Adobe/)
% \usepackage{NotoSansSC_external}
% \usepackage{NotoSerifCJKsc_external}
% \usepackage{zh_CN-Adobefonts_internal} % Simplified Chinese Support using system fonts
\usepackage{linespacing_fix} % disable extra space before next section
\usepackage{cite}

\begin{document}
\pagenumbering{gobble} % suppress displaying page number

\name{赵之航}

\basicInfo{
  \email{2018311178@email.cufe.edu.cn} \textperiodcentered\ 
  \phone{+86~186~3689~3217} \textperiodcentered\ 
  \homepage{https://lucaszhao22.github.io/}}

\section{\faGraduationCap\  教育背景}
\datedsubsection{\textbf{中央财经大学}, 北京}{2018 -- 至今}
\textit{在读本科生}\ 国民经济管理
{\small
    \begin{itemize}
      \item{经济学原理(1)(94),经济学原理(2)(90),计量经济学导论(双语)(89),数理经济学(98),中级宏观经济学(双语)(93),
      中级微观经济学(双语)(86),实变函数(94)等}
    \end{itemize}
    }

\section{\faUsers\ 项目经历}
\datedsubsection{\textbf{招生计划研究}}{2019年12月 -- 2020年2月}
\role{数据录入}{郭冬梅教授}
将招生指南中山西省的招生数据整理录入Excel
\begin{itemize}
  \item OCR扫描将照片转成文档,python程序初步处理将文档转录为Excel表格,使用xlrd,openpyxl,pandas模块,利用Excel宏将表格按需求拆分
  \item 对比人工录入,效率提升近10倍,使用本工作流可以一次性处理3000条以上的数据,为排错方便,建议一次处理的数据条目在1000到2000条左右
\end{itemize}

\datedsubsection{\textbf{疫情防控通自动打卡}}{2020年6月}
\role{Python}{个人项目,基于GitHub中的开源程序开发}
\begin{onehalfspacing}
  自动化运行的填写疫情防控通脚本,托管于腾讯云函数
\begin{itemize}
  \item 使用Fiddler抓取登录的cookie进行模拟网页登录,存储post的信息,利用对应的api每日填报
  \item 该项目使学生免于每日重复地填报,仅在有必要时进行手动更改上报
\end{itemize}
\end{onehalfspacing}

\datedsubsection{\textbf{个人网站}}{2020年8月}
\role{Markdown}{个人项目}
\begin{onehalfspacing}
采用hexo框架,主题为butterfly,并进行一些定制化,托管于GitHub。
\end{onehalfspacing}

% Reference Test
%\datedsubsection{\textbf{Paper Title\cite{zaharia2012resilient}}}{May. 2015}
%An xxx optimized for xxx\cite{verma2015large}
%\begin{itemize}
%  \item main contribution
%\end{itemize}

\section{\faCogs\ 个人技能}
% increase linespacing [parsep=0.5ex]
\begin{itemize}[parsep=0.5ex]
  \item 编程语言: Python、 \LaTeX\ 、Markdown、Git、Stata,可熟练使用office,\textbf{材料制作能力较强,样式美观完善}
  \item 英语:四级620,六级555,托福90(\textbf{听力满分}),\textbf{有较强的听译及翻译能力},在哔哩哔哩的个人账号上发布有3集独立制作的网剧翻译作品,\textbf{口语流利}                
  \item 专业知识:专业课成绩良好,\textbf{有较强的数理能力}
  \item 具有一定的论文写作经验,\textbf{研究能力较强},在2020年全国大学生数学建模比赛中作为队长独立完成B题中三道问题的模型建模和论文写作,为后续具体问题的求解建立了基本方法,
  使用 \LaTeX\ 进行排版,协调组员完成论文
\end{itemize}

\section{\faTrophy\ 获奖情况}
\datedline{北京市第三十届大学生数学竞赛(经管类), \textit{二等奖}}{2019 年12 月}
\datedline{第十一届全国大学生数学竞赛(非数学类), \textit{二等奖}}{2019 年11 月}
\datedline{2020年全国大学生数学建模比赛, \textit{北京市一等奖}}{2020 年11 月}
\datedline{泸州老窖奖励基金}{2019 年6 月}

\section{\faInfo\ 其他}
% increase linespacing [parsep=0.5ex]
\begin{itemize}[parsep=0.5ex]
  %\item 个人博客: https://lucaszhao22.github.io/
  \item \textbf{有较强的自学能力和信息检索能力},对待工作认真负责,乐于协作,耐心细致,追求完美
  \item 有较强的抗压能力,接受加班
\end{itemize}

%% Reference
%\newpage
%\bibliographystyle{IEEETran}
%\bibliography{mycite}
\end{document}
