\documentclass[lang=cn]{elegantpaper}

% 标题信息
\title{OTT服务商业模式浅析\\ \begin{large} ——以Netflix为例 \end{large} }
\author{赵之航 2018311178}
\institute{经济学院\,国民经济管理18}
\date{\zhtoday}

\begin{document}

\maketitle

\begin{abstract}
在这里写摘要。
\keywords{关键词1,关键词2}
\end{abstract}

\section{引言}

OTT服务,即Over-the-top media services,这是一种直接向观众提供的互联网流媒体服务。
相对于传统媒体,OTT不需要申领牌照,拥有很大的内容自主权,因此在不同的国家,OTT服务的政策环境也不尽相同。
总体而言,北美等地区的政策环境较为宽松,审核不严格,而国内则审核较为严格。根据Statista发布的数据,我们可以看出北美与中国是全球前两大OTT市场。
截至2020年1月份,以每100个家庭拥有的OTT服务项目数量计算,美国以164项OTT服务高居全球榜首,加拿大则有132项,使得北美成为全球最大的OTT市场,中国以68项OTT服务紧随其后。
北美作为全球最大的OTT市场,孕育出了Netflix这样全球市场份额高达26.58%的超级巨头,对北美乃至全球的OTT服务商都有深远的影响。据估计,全世界37%的互联网用户都在使用Netflix。
Netflix在美国和加拿大吸引了超过2300万用户,并且最高可以流式传输支持4K、HDR、杜比视界、杜比全景声的视频,传输视频的平均比特率高达3.6 Mbps。
Netflix是美国宽带流量消耗的最大来源,峰值时间占据29.7%的下行流量。

Netflix的商业模式毫无疑问是成功的,也引来了同行的效仿,对其商业模式的研究亦成果丰硕。本文将从Netflix在流媒体世代的转型与起步开始,
分别探究Netflix在技术、盈利模式、全球布局等方面的优势所在,深入了解Netflix的创新之处,并分析这些创新为Netflix的成功起到了何种作用。
在分析Netflix商业模式的基础上,我们将放眼其他OTT服务商,了解这些厂商的特色以及与Netflix的相同与不同之处,并在其中挑选竞争厂商分析Netflix对这些厂商的影响与面对这些厂商的优势和不足。
接着我们回到国内市场,对国内类似厂商进行观察与分析,探究Netflix对他们的影响以及国内厂商做出的本地化改进。
最后,我们在可预见的未来展望Netflix可能的发展状况,讨论其未来可能遇到的问题,并针对每个问题给出可能的解答。

\section{Netflix转型背景}

Netflix成立于1997年,公司的主营业务是出租音像制品——即DVD。

\section{OTT服务的技术背景}

\section{Netflix盈利模式分析}

\section{Netflix商业模式创新}

\section{Netflix全球布局现状}

\section{其他OTT服务商}

\subsection{强有力的竞争厂商}

\subsection{国内类似厂商}

\section{展望未来:Netflix的优势与不足}

\section{结论}
这是结论部分。
% 在参考文献部分显示未引用的文献

\nocite{ref1, ref2}

% 生成参考文献
\bibliography{wpref}

\end{document}