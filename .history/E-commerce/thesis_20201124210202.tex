\documentclass[lang=cn]{elegantpaper}

% 标题信息
\title{OTT服务商业模式浅析\\ \begin{large} ——以Netflix为例 \end{large} }
\author{赵之航 2018311178}
\institute{经济学院\,国民经济管理18}
\date{\zhtoday}

\begin{document}

\maketitle

\begin{abstract}
在这里写摘要。
\keywords{关键词1,关键词2}
\end{abstract}

\tableofcontents
\newpage

\section{引言}

OTT服务,即Over-the-top media services,这是一种直接向观众提供的互联网流媒体服务。
相对于传统媒体,OTT不需要申领牌照,拥有很大的内容自主权,因此在不同的国家,OTT服务的政策环境也不尽相同。
总体而言,北美等地区的政策环境较为宽松,审核不严格,而国内则审核较为严格。根据Statista发布的数据,我们可以看出北美与中国是全球前两大OTT市场。
截至2020年1月份,以每100个家庭拥有的OTT服务项目数量计算,美国以164项OTT服务高居全球榜首,加拿大则有132项,使得北美成为全球最大的OTT市场,中国以68项OTT服务紧随其后。
北美作为全球最大的OTT市场,孕育出了Netflix这样全球市场份额高达26.58%的超级巨头,对北美乃至全球的OTT服务商都有深远的影响。据估计,全世界37%的互联网用户都在使用Netflix。
Netflix在美国和加拿大吸引了超过2300万用户,并且最高可以流式传输支持4K、HDR、杜比视界、杜比全景声的视频,传输视频的平均比特率高达3.6 Mbps。
Netflix是美国宽带流量消耗的最大来源,峰值时间占据29.7%的下行流量。

Netflix的商业模式毫无疑问是成功的,也引来了同行的效仿,对其商业模式的研究亦成果丰硕。本文将从Netflix在流媒体世代的转型与起步开始,
分别探究Netflix在技术、盈利模式、全球布局等方面的优势所在,深入了解Netflix的创新之处,并分析这些创新为Netflix的成功起到了何种作用。
在分析Netflix商业模式的基础上,我们将放眼其他OTT服务商,了解这些厂商的特色以及与Netflix的相同与不同之处,并在其中挑选竞争厂商分析Netflix对这些厂商的影响与面对这些厂商的优势和不足。
接着我们回到国内市场,对国内类似厂商进行观察与分析,探究Netflix对他们的影响以及国内厂商做出的本地化改进。
最后,我们在可预见的未来展望Netflix可能的发展状况,讨论其未来可能遇到的问题,并针对每个问题给出可能的解答。

\section{Netflix转型背景}

Netflix成立于1997年,公司的主营业务是出租音像制品——即DVD,而在这一领域,Blockbuster是当时的头部。Netflix作为新秀的破局方法是推出了一个大约900个内容的视频库,用户在其网站上订购DVD,
公司将其邮寄给用户,用户观影完毕再邮寄回公司。DVD最长租赁期为7天,租金最初只需50美分一个。当时的Netflix只是一种更便捷且便宜的租电影方式,人们都将其视为一家新兴的经销商。
1999年,Netflix推出了全新的订阅模式,初始定价15.95美元,Netflix会员每次可以免邮租赁4部电影,不限归还日期,想要租借新的影碟,就要归还正在租赁的影碟。
这相较于传统的租赁服务是一次巨大的创新,没有逾期,没有罚金,平均到每一张DVD上的租金更低,这对于消费者有着极大的吸引力。这样的租赁方式,不仅节省了用户的时间成本,而且节省了用户的邮资等
费用,并且能够使用户可选择影片的范围大大增加,无疑是更先进的。得益于先进的租赁方式,Netflix在2000-2003年实现了持续增长,但尽管用户和收入增长都非常可观,Netflix仍在亏损,
仅2002年第一季度就亏损了450万美元。这是由于Netflix过高的业务费用导致的。好在其用户的增长可观且持续,这样的情况并未持续太久,Netflix在2006年订阅用户超600万,盈利利润超8000万美元。

尽管已经成为北美最大的DVD租赁公司之一,但Netflix并没有像他的竞争对手一样只专注于眼前的业务。2007年,经过两年的滞销后,DVD市场萎缩了4.5%,这是自10年前推出这种格式以来,
DVD销量首次同比下降。尽管Netflix的DVD租赁业务增长可观且有盈利,Netflix仍将目光投向未来,思考企业转型的问题。YouTube的崛起让Netflix看到了流媒体的潜力,既然YouTube主打用户原创内容,
Netflix便将目光投向了自己的影视租赁业务。Netflix希望通过DVD租赁业务进一步扩大用户数量,然后再将用户转移到在线流媒体服务,这在当时是极为激进且颇受争议的策略。从现在的角度来看,转型流媒体是Netflix
当时自身已有业务的延伸,更是极具前瞻性的成功策略,但在2007年,流媒体技术刚刚起步,最快的宽带连接也不能很好地传输高分辨率视频,这意味着视频质量会比DVD差。
尽管如此,Netflix还是决定推出第一款流媒体产品“Watch Now”,免费包含在已有的订阅计划内。Watch Now只兼容Windows操作系统,用户需要下载插件才能在Internet Explorer上收看视频,总体而言体验并不优秀。
不过情况在次年得到了极大的改善。2008年,Netflix宣布与美国有线电视公司Starz建立合作关系,Starz为Netflix提供了2500多部电影和电视节目的版权。
Netflix的流媒体模式获得了成功,从2007年推出Watch Now到2011年底,Netflix的用户数量从600万增加到2300万,仅在四年内就增加了283%。

Netflix流媒体业务的成功使得行业中各个部分的企业都想参与进来获取更大的利益。2012年,Starz取消了与Netflix的许可协议,这使得Netflix的流媒体影视库中瞬间减少了上千部电影。
Netflix花费在内容版权上的费用越来越高,这使得企业思考自己的第二次转型——自制内容。这一次转型的创新是前所未有的,真正建立起了Netflix的商业模式护城河,日后也引来了同行业竞争对手的效仿。
我们将在第五节详细分析。

\section{OTT服务的技术背景}

所有OTT服务提供商首要解决的问题是——如何将高质量的视频内容无损传输给用户。这对于厂商的技术要求很高,我们将以Netflix为例,探究OTT服务的技术力需求。

为了高速传输高质量的视频,Netflix的串流平台架构主要由以下四个部分组成:Netflix数据中心,亚马逊云,CDN和播放器。

\section{Netflix盈利模式分析}

\section{Netflix商业模式创新}

\section{Netflix全球布局现状}

\section{其他OTT服务商}

\subsection{强有力的竞争厂商}

\subsection{国内类似厂商}

\section{展望未来:Netflix的优势与不足}

\section{结论}
这是结论部分。
% 在参考文献部分显示未引用的文献

\nocite{ref1, ref2}

% 生成参考文献
\bibliography{wpref}

\end{document}