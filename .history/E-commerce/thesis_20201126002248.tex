\documentclass[lang=cn]{elegantpaper}

\usepackage{threeparttable}

% 标题信息
\title{OTT服务商业模式浅析\\ \begin{large} ——以Netflix为例 \end{large} }
\author{赵之航 2018311178}
\institute{经济学院\,国民经济管理18}
\date{\zhtoday}

\begin{document}

\maketitle

\begin{abstract}
在这里写摘要。
\keywords{关键词1,关键词2}
\end{abstract}

\tableofcontents
\newpage

\section{引言}

OTT服务,即Over-the-top media services,这是一种直接向观众提供的互联网流媒体服务。
相对于传统媒体,OTT不需要申领牌照,拥有很大的内容自主权,因此在不同的国家,OTT服务的政策环境也不尽相同。
总体而言,北美等地区的政策环境较为宽松,审核不严格,而国内则审核较为严格。根据Statista发布的数据,我们可以看出北美与中国是全球前两大OTT市场。
截至2020年1月份,以每100个家庭拥有的OTT服务项目数量计算,美国以164项OTT服务高居全球榜首,加拿大则有132项,使得北美成为全球最大的OTT市场,中国以68项OTT服务紧随其后。
北美作为全球最大的OTT市场,孕育出了Netflix这样全球市场份额高达26.58%的超级巨头,对北美乃至全球的OTT服务商都有深远的影响。据估计,全世界37%的互联网用户都在使用Netflix。
Netflix在美国和加拿大吸引了超过2300万用户,并且最高可以流式传输支持4K、HDR、杜比视界、杜比全景声的视频,传输视频的平均比特率高达3.6 Mbps。
Netflix是美国宽带流量消耗的最大来源,峰值时间占据29.7%的下行流量。

Netflix的商业模式毫无疑问是成功的,也引来了同行的效仿,对其商业模式的研究亦成果丰硕。本文将从Netflix在流媒体世代的转型与起步开始,
分别探究Netflix在技术、盈利模式、全球布局等方面的优势所在,深入了解Netflix的创新之处,并分析这些创新为Netflix的成功起到了何种作用。
在分析Netflix商业模式的基础上,我们将放眼其他OTT服务商,了解这些厂商的特色以及与Netflix的相同与不同之处,并在其中挑选竞争厂商分析Netflix对这些厂商的影响与面对这些厂商的优势和不足。
接着我们回到国内市场,对国内类似厂商进行观察与分析,探究Netflix对他们的影响以及国内厂商做出的本地化改进。
最后,我们在可预见的未来展望Netflix可能的发展状况,讨论其未来可能遇到的问题,并针对每个问题给出可能的解答。

\section{Netflix转型背景}

Netflix成立于1997年,公司的主营业务是出租音像制品——即DVD,而在这一领域,Blockbuster是当时的头部。Netflix作为新秀的破局方法是推出了一个大约900个内容的视频库,用户在其网站上订购DVD,
公司将其邮寄给用户,用户观影完毕再邮寄回公司。DVD最长租赁期为7天,租金最初只需50美分一个。当时的Netflix只是一种更便捷且便宜的租电影方式,人们都将其视为一家新兴的经销商。
1999年,Netflix推出了全新的订阅模式,初始定价15.95美元,Netflix会员每次可以免邮租赁4部电影,不限归还日期,想要租借新的影碟,就要归还正在租赁的影碟。
这相较于传统的租赁服务是一次巨大的创新,没有逾期,没有罚金,平均到每一张DVD上的租金更低,这对于消费者有着极大的吸引力。这样的租赁方式,不仅节省了用户的时间成本,而且节省了用户的邮资等
费用,并且能够使用户可选择影片的范围大大增加,无疑是更先进的。得益于先进的租赁方式,Netflix在2000-2003年实现了持续增长,但尽管用户和收入增长都非常可观,Netflix仍在亏损,
仅2002年第一季度就亏损了450万美元。这是由于Netflix过高的业务费用导致的。好在其用户的增长可观且持续,这样的情况并未持续太久,Netflix在2006年订阅用户超600万,盈利利润超8000万美元。

尽管已经成为北美最大的DVD租赁公司之一,但Netflix并没有像他的竞争对手一样只专注于眼前的业务。2007年,经过两年的滞销后,DVD市场萎缩了4.5%,这是自10年前推出这种格式以来,
DVD销量首次同比下降。尽管Netflix的DVD租赁业务增长可观且有盈利,Netflix仍将目光投向未来,思考企业转型的问题。YouTube的崛起让Netflix看到了流媒体的潜力,既然YouTube主打用户原创内容,
Netflix便将目光投向了自己的影视租赁业务。Netflix希望通过DVD租赁业务进一步扩大用户数量,然后再将用户转移到在线流媒体服务,这在当时是极为激进且颇受争议的策略。从现在的角度来看,转型流媒体是Netflix
当时自身已有业务的延伸,更是极具前瞻性的成功策略,但在2007年,流媒体技术刚刚起步,最快的宽带连接也不能很好地传输高分辨率视频,这意味着视频质量会比DVD差。
尽管如此,Netflix还是决定推出第一款流媒体产品“Watch Now”,免费包含在已有的订阅计划内。Watch Now只兼容Windows操作系统,用户需要下载插件才能在Internet Explorer上收看视频,总体而言体验并不优秀。
不过情况在次年得到了极大的改善。2008年,Netflix宣布与美国有线电视公司Starz建立合作关系,Starz为Netflix提供了2500多部电影和电视节目的版权。
Netflix的流媒体模式获得了成功,从2007年推出Watch Now到2011年底,Netflix的用户数量从600万增加到2300万,仅在四年内就增加了283%。

Netflix流媒体业务的成功使得行业中各个部分的企业都想参与进来获取更大的利益。2012年,Starz取消了与Netflix的许可协议,这使得Netflix的流媒体影视库中瞬间减少了上千部电影。
Netflix花费在内容版权上的费用越来越高,这使得企业思考自己的第二次转型——自制内容。这一次转型的创新是前所未有的,真正建立起了Netflix的商业模式护城河,日后也引来了同行业竞争对手的效仿。
我们将在第五节详细分析。

\section{OTT服务的技术背景}

所有OTT服务提供商首要解决的问题是——如何将高质量的视频内容无损传输给用户。这对于厂商的技术要求很高,我们将以Netflix为例,探究OTT服务的技术力需求。

为了高速传输高质量的视频,Netflix的串流平台架构主要由以下四个部分组成:Netflix数据中心,亚马逊云,CDN和播放器。
\begin{figure}[!h]
    \centering
    \includegraphics[width=.6\textwidth]{tech}
    \caption{Netflix架构示意图}
\end{figure}
\begin{enumerate}
    \item Netflix数据中心\par
    \setlength{\parindent}{2em}www.netflix.com指向是Netflix的IP地址,该服务器主要的功能有注册账户、采集付款信息、检测是否登录。在用户观影
    时,客户端不会与服务器进行任何交互。
    \item 亚马逊云\par
    \setlength{\parindent}{2em}除了由Netflix托管的www.netflix.com之外,其余大多都由亚马逊云提供服务,服务内容包括视频存储,DRM加密,CDN,日志记录,用户登录等。
    \item CDN \par
    \setlength{\parindent}{2em}CDN全称Content Delivery Network,即内容分发网络,可以形象地理解为“网络加速器”。Netflix将受DRM加密并编码好的视频从亚马逊云复制到多个CDN,再从CDN将视频最终交付到用户。
    Netflix采用了三种CDN:Akamai,LimeLight和Level-3。同一质量同一视频,从3个CDN传输的编码内容均相同。当开始传输视频内容时,Netflix会对三个CDN的传输速度测速并排序,并选择速度最快的一个CDN进行串流,
    但当该CDN与客户端的连接速率下降时,CDN并不会切换,Netflix会降低传输视频的质量,直到速率下降至100Kpbs,才会切换至第二个CDN,即Netflix倾向于在速率下降时降低视频质量而不切换CDN。
    \begin{table}[!htbp]
        \centering
        \begin{threeparttable}
            \begin{tabular}{cc}
                \toprule[1.5pt]
                \makebox[0.3\textwidth][c]{网址}	&  \makebox[0.4\textwidth][c]{服务提供商} \\
                \midrule[1pt]
                www.netflix.com                         &Netflix\\
                signup.netflix.com                      &Amazon\\
                movies.netflix.com                      &Amazon\\
                agmoviecontrol.netflix.com              &Amazon\\
                nflx.i.87f50a04.x.lcdn.nflximg.com      &Level 3\\
                netflix-753.vo.llnwd.net                &Limelight\\
                netflix753.as.nflximg.com.edgesuite.net &Akamai\\
                \bottomrule[1.5pt]
            \end{tabular}
        \end{threeparttable}
    \end{table}
    \item 播放器\par
    \setlength{\parindent}{2em}在桌面浏览器上,Netflix使用Silverlight进行视频的下载、解码和播放。Netflix在全平台(手机、电视盒子等)都使用HTML5播放器,Netflix也是第一批弃用flash转向HTML5的企业之一。
\end{enumerate}

Netflix还使用了DASH协议进行视频传输,即将视频分成不同的块进行编码,每个块分别编码多个不同质量的视频,在传输时,一次传输一个块,并依据客户端与CDN之间的连接速率决定下一个块传输的视频质量。
DASH技术的应用极大地减少了网络不稳定带来的播放卡顿。

高画质是Netflix主打的功能之一,Netflix也是推广高画质标准的推动者之一,但另一方面,不是每个用户都有可以流畅传输高画质视频的网络条件,因此如何用尽量少的带宽来传输尽量高的画质是Netflix需要解决的问题。
对此,Netflix对其内容库的编码进行了优化,这一技术便是“动态优化器”。动态优化器会利用AI算法分析一个视频里每一帧的画面,并对不同的画面分配不同的码率上限。例如画面中的黑屏或其他较为简单的场景分配较低的
码率,而画面更复杂(例如动作戏、追车戏、爆炸场面)的部分则分配更高的码率。不同类型的作品也会有不同的算法来进行优化。动态优化器在不降低画面质量的前提下,大大缩小了视频的体积,可为用户节省约20%的带宽。

从Netflix的技术应用我们可以看出,OTT服务提供商必须先将最基本的传输视频体验做好,这是OTT服务最底层、最需要的功能。Netflix视频观看体验上多年的技术耕耘,也是Netflix自身业务的一道护城河。

\section{Netflix盈利模式分析}

Netflix的收入来源较为单一,其收入的98%来自于提供视频点播服务收取的订阅会员费,其余收入则来自美国国内单一费率邮寄DVD租赁业务的收入。由此我们看出,Netflix是一家典型的SaaS(软件即服务)
模式的公司。SaaS模式是一种通过网络提供软件服务的公司,即用户不再购买软件,而是通过公司提供的网络服务来享受软件的功能,用户无需学习如何使用、维护软件。具体到Netflix的例子则是,由Netflix
提供数量庞大的影视库,用户无需购买或租赁实体的DVD,并整理由实体DVD组成的个人影视库。用户只需缴纳订阅费,便可得到访问并观看Netflix影视库的权限,免去了购买、整理DVD的时间成本。

SaaS模式与传统模式存在差异,主要有如下几点区别:
\begin{itemize}
    \item SaaS模式需要用户以一定的时间周期支付订阅费用,通常这个周期是一个月/一个季度/一年,供应商一般会在第一次付费之前提供天数不等的试用;传统模式则是一次性买断。
    \item SaaS模式不需要用户对软件服务进行安装或配置,直接通过网络即可访问服务;传统模式则需要用户进行一定的安装或配置(例如配置蓝光播放器等)。
    \item SaaS模式会自动更新服务(影片上新不需要购买),而传统模式需要自行更新(影片上新需要购买)。
    \item 相比传统模式,SaaS模式的数据安全性更好(影片不会丢失)。
    \item SaaS模式中,提供商承担了维护费用,对用户更加友好;传统模式则需要自行维护,较为麻烦(DVD碟片需要妥善放置)
    \item SaaS模式提供的服务可以一次从电脑、电视、手机等多端享受,传统模式则只能一次通过一处享受。
\end{itemize}

Netflix自身的关键业务——影片租赁——是非常适合SaaS模式的。普通消费者观看电影的途径有限,在Netflix转型之前只有两种:影院观看、购买或租赁DVD。影院观看票价较贵,只能在上映期间观看,不够便捷;
购买或租赁DVD也有时间成本,缺乏快速接触到大量影视库的途径。流媒体模式则克服了上述两种观影方式的缺点,有着快速、大量、便宜的优势,消费者按需订阅即可,在不观看影片的时间段可以取消订阅,不需要
为了观影花费观影之外的时间(前往电影院的通勤时间、挑选/邮寄DVD的时间)。据此分析我们可以总结出SaaS模式相较于传统模式的优点:
\begin{itemize}
    \item 低成本。用户只需花费订阅费便可以在订阅期限内观看Netflix庞大影视库内的全部电影。
    \item 易于使用。登录Netflix便可以开始观影,不需要用户学习如何配置播放器、解码器、渲染器等,减少了学习成本,节省了用户的时间。
    \item 提供集成。Netflix与Roku、Nvidia等公司的电视盒子有合作,软件集成在硬件内。
    \item 不需要担心升级。Netflix会负责影片上新、技术升级的工作。
    \item 定价灵活。Netflix有三档会员,定价不同,对视频质量、可同时观看的屏幕分别做了限制,用户可按需选择适合自己的订阅模式。
\end{itemize}

当然,SaaS模式也有其劣势,例如数据安全风险(账号有被窃取的可能)、有限的集成功能(集成Netflix的硬件需要通过Netflix的认证)等,但总体而言,SaaS模式的优势远超其局限性,且局限性可以通过技术
手段的更新来克服。

2019年Netflix的财报显示,公司全年总收入201.56 亿美元,其中会员费收入为198.59 亿美元,这里有92.43 亿美元来自美国地区,有106.16 亿美元来自除美国外其他地区。Netflix用户的付费意愿远超其余流媒体服务商。
究其原因,是Netflix的核心资源足够吸引人。Netflix的影片库数量庞大,其中更是有独家制作的口碑影视,在Netflix独播,这些影视资源使得Netflix的用户付费意愿持续上升。Netflix也在加大对第一方制作影视的投入,
既可以完全掌握版权,也可以巩固自身的用户群。但同时我们也要注意到,第一方制作内容的成本是极高的,这也意味着Netflix必须维持现有的高增长才能保持未来持续的内容输出,巩固自身的用户群。Netflix的内容成本
占比从2013起一直保持在50%左右,加上客服、流量、支付平台手续费、营销以及研发等成本,Netflix虽然在2019年年收入实现了盈利,但长期来看Netflix仍然需要举债才能覆盖自身的业务成本,即Netflix实现了正现金流,
利润仍然是负数。面对透支未来收入用来制作内容的局面,如何保持自身的增长,使得未来的收入可以覆盖当年的摊销,是Netflix盈利模式需要关注的重点。

\section{Netflix商业模式创新}

\section{Netflix全球布局现状}

\section{其他OTT服务商}

\subsection{强有力的竞争厂商}

\subsection{国内类似厂商}

\section{展望未来:Netflix的优势与不足}

\section{结论}
这是结论部分。
% 在参考文献部分显示未引用的文献

\nocite{ref1, ref2}

% 生成参考文献
\bibliography{wpref}

\end{document}