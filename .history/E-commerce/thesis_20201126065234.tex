\documentclass[lang=cn]{elegantpaper}

\usepackage{threeparttable}

% 标题信息
\title{OTT服务商业模式浅析\\ \begin{large} ——以Netflix为例 \end{large} }
\author{赵之航 2018311178}
\institute{经济学院\,国民经济管理18}
\date{\zhtoday}

\begin{document}

\maketitle

\begin{abstract}
在这里写摘要。
\keywords{关键词1,关键词2}
\end{abstract}

\tableofcontents
\newpage

\section{引言}

OTT服务,即Over-the-top media services,这是一种直接向观众提供的互联网流媒体服务。
相对于传统媒体,OTT不需要申领牌照,拥有很大的内容自主权,因此在不同的国家,OTT服务的政策环境也不尽相同。
总体而言,北美等地区的政策环境较为宽松,审核不严格,而国内则审核较为严格。根据Statista发布的数据,我们可以看出北美与中国是全球前两大OTT市场。
截至2020年1月份,以每100个家庭拥有的OTT服务项目数量计算,美国以164项OTT服务高居全球榜首,加拿大则有132项,使得北美成为全球最大的OTT市场,中国以68项OTT服务紧随其后。
北美作为全球最大的OTT市场,孕育出了Netflix这样全球市场份额高达26.58%的超级巨头,对北美乃至全球的OTT服务商都有深远的影响。据估计,全世界37%的互联网用户都在使用Netflix。
Netflix在美国和加拿大吸引了超过2300万用户,并且最高可以流式传输支持4K、HDR、杜比视界、杜比全景声的视频,传输视频的平均比特率高达3.6 Mbps。
Netflix是美国宽带流量消耗的最大来源,峰值时间占据29.7%的下行流量。

Netflix的商业模式毫无疑问是成功的,也引来了同行的效仿,对其商业模式的研究亦成果丰硕。本文将从Netflix在流媒体世代的转型与起步开始,
分别探究Netflix在技术、盈利模式、全球布局等方面的优势所在,深入了解Netflix的创新之处,并分析这些创新为Netflix的成功起到了何种作用。
在分析Netflix商业模式的基础上,我们将放眼其他OTT服务商,了解这些厂商的特色以及与Netflix的相同与不同之处,并在其中挑选竞争厂商分析Netflix对这些厂商的影响与面对这些厂商的优势和不足。
接着我们回到国内市场,对国内类似厂商进行观察与分析,探究Netflix对他们的影响以及国内厂商做出的本地化改进。
最后,我们在可预见的未来展望Netflix可能的发展状况,讨论其未来可能遇到的问题,并针对每个问题给出可能的解答。

\section{Netflix转型背景}

Netflix成立于1997年,公司的主营业务是出租音像制品——即DVD,而在这一领域,Blockbuster是当时的头部。Netflix作为新秀的破局方法是推出了一个大约900个内容的视频库,用户在其网站上订购DVD,
公司将其邮寄给用户,用户观影完毕再邮寄回公司。DVD最长租赁期为7天,租金最初只需50美分一个。当时的Netflix只是一种更便捷且便宜的租电影方式,人们都将其视为一家新兴的经销商。
1999年,Netflix推出了全新的订阅模式,初始定价15.95美元,Netflix会员每次可以免邮租赁4部电影,不限归还日期,想要租借新的影碟,就要归还正在租赁的影碟。
这相较于传统的租赁服务是一次巨大的创新,没有逾期,没有罚金,平均到每一张DVD上的租金更低,这对于消费者有着极大的吸引力。这样的租赁方式,不仅节省了用户的时间成本,而且节省了用户的邮资等
费用,并且能够使用户可选择影片的范围大大增加,无疑是更先进的。得益于先进的租赁方式,Netflix在2000-2003年实现了持续增长,但尽管用户和收入增长都非常可观,Netflix仍在亏损,
仅2002年第一季度就亏损了450万美元。这是由于Netflix过高的业务费用导致的。好在其用户的增长可观且持续,这样的情况并未持续太久,Netflix在2006年订阅用户超600万,盈利利润超8000万美元。

尽管已经成为北美最大的DVD租赁公司之一,但Netflix并没有像他的竞争对手一样只专注于眼前的业务。2007年,经过两年的滞销后,DVD市场萎缩了4.5%,这是自10年前推出这种格式以来,
DVD销量首次同比下降。尽管Netflix的DVD租赁业务增长可观且有盈利,Netflix仍将目光投向未来,思考企业转型的问题。YouTube的崛起让Netflix看到了流媒体的潜力,既然YouTube主打用户原创内容,
Netflix便将目光投向了自己的影视租赁业务。Netflix希望通过DVD租赁业务进一步扩大用户数量,然后再将用户转移到在线流媒体服务,这在当时是极为激进且颇受争议的策略。从现在的角度来看,转型流媒体是Netflix
当时自身已有业务的延伸,更是极具前瞻性的成功策略,但在2007年,流媒体技术刚刚起步,最快的宽带连接也不能很好地传输高分辨率视频,这意味着视频质量会比DVD差。
尽管如此,Netflix还是决定推出第一款流媒体产品“Watch Now”,免费包含在已有的订阅计划内。Watch Now只兼容Windows操作系统,用户需要下载插件才能在Internet Explorer上收看视频,总体而言体验并不优秀。
不过情况在次年得到了极大的改善。2008年,Netflix宣布与美国有线电视公司Starz建立合作关系,Starz为Netflix提供了2500多部电影和电视节目的版权。
Netflix的流媒体模式获得了成功,从2007年推出Watch Now到2011年底,Netflix的用户数量从600万增加到2300万,仅在四年内就增加了283%。

Netflix流媒体业务的成功使得行业中各个部分的企业都想参与进来获取更大的利益。2012年,Starz取消了与Netflix的许可协议,这使得Netflix的流媒体影视库中瞬间减少了上千部电影。
Netflix花费在内容版权上的费用越来越高,这使得企业思考自己的第二次转型——自制内容。这一次转型的创新是前所未有的,真正建立起了Netflix的商业模式护城河,日后也引来了同行业竞争对手的效仿。
我们将在第五节详细分析。

\section{OTT服务的技术背景}

所有OTT服务提供商首要解决的问题是——如何将高质量的视频内容无损传输给用户。这对于厂商的技术要求很高,我们将以Netflix为例,探究OTT服务的技术力需求。

为了高速传输高质量的视频,Netflix的串流平台架构主要由以下四个部分组成:Netflix数据中心,亚马逊云,CDN和播放器。
\begin{figure}[!h]
    \centering
    \includegraphics[width=.6\textwidth]{tech}
    \caption{Netflix架构示意图}
\end{figure}
\begin{enumerate}
    \item Netflix数据中心\par
    \setlength{\parindent}{2em}www.netflix.com指向是Netflix的IP地址,该服务器主要的功能有注册账户、采集付款信息、检测是否登录。在用户观影
    时,客户端不会与服务器进行任何交互。
    \item 亚马逊云\par
    \setlength{\parindent}{2em}除了由Netflix托管的www.netflix.com之外,其余大多都由亚马逊云提供服务,服务内容包括视频存储,DRM加密,CDN,日志记录,用户登录等。
    \item CDN \par
    \setlength{\parindent}{2em}CDN全称Content Delivery Network,即内容分发网络,可以形象地理解为“网络加速器”。Netflix将受DRM加密并编码好的视频从亚马逊云复制到多个CDN,再从CDN将视频最终交付到用户。
    Netflix采用了三种CDN:Akamai,LimeLight和Level-3。同一质量同一视频,从3个CDN传输的编码内容均相同。当开始传输视频内容时,Netflix会对三个CDN的传输速度测速并排序,并选择速度最快的一个CDN进行串流,
    但当该CDN与客户端的连接速率下降时,CDN并不会切换,Netflix会降低传输视频的质量,直到速率下降至100Kpbs,才会切换至第二个CDN,即Netflix倾向于在速率下降时降低视频质量而不切换CDN。
    \begin{table}[!htbp]
        \centering
        \begin{threeparttable}
            \begin{tabular}{cc}
                \toprule[1.5pt]
                \makebox[0.3\textwidth][c]{网址}	&  \makebox[0.4\textwidth][c]{服务提供商} \\
                \midrule[1pt]
                www.netflix.com                         &Netflix\\
                signup.netflix.com                      &Amazon\\
                movies.netflix.com                      &Amazon\\
                agmoviecontrol.netflix.com              &Amazon\\
                nflx.i.87f50a04.x.lcdn.nflximg.com      &Level 3\\
                netflix-753.vo.llnwd.net                &Limelight\\
                netflix753.as.nflximg.com.edgesuite.net &Akamai\\
                \bottomrule[1.5pt]
            \end{tabular}
        \end{threeparttable}
    \end{table}
    \item 播放器\par
    \setlength{\parindent}{2em}在桌面浏览器上,Netflix使用Silverlight进行视频的下载、解码和播放。Netflix在全平台(手机、电视盒子等)都使用HTML5播放器,Netflix也是第一批弃用flash转向HTML5的企业之一。
\end{enumerate}

Netflix还使用了DASH协议进行视频传输,即将视频分成不同的块进行编码,每个块分别编码多个不同质量的视频,在传输时,一次传输一个块,并依据客户端与CDN之间的连接速率决定下一个块传输的视频质量。
DASH技术的应用极大地减少了网络不稳定带来的播放卡顿。

高画质是Netflix主打的功能之一,Netflix也是推广高画质标准的推动者之一,但另一方面,不是每个用户都有可以流畅传输高画质视频的网络条件,因此如何用尽量少的带宽来传输尽量高的画质是Netflix需要解决的问题。
对此,Netflix对其内容库的编码进行了优化,这一技术便是“动态优化器”。动态优化器会利用AI算法分析一个视频里每一帧的画面,并对不同的画面分配不同的码率上限。例如画面中的黑屏或其他较为简单的场景分配较低的
码率,而画面更复杂(例如动作戏、追车戏、爆炸场面)的部分则分配更高的码率。不同类型的作品也会有不同的算法来进行优化。动态优化器在不降低画面质量的前提下,大大缩小了视频的体积,可为用户节省约20%的带宽。

从Netflix的技术应用我们可以看出,OTT服务提供商必须先将最基本的传输视频体验做好,这是OTT服务最底层、最需要的功能。Netflix视频观看体验上多年的技术耕耘,也是Netflix自身业务的一道护城河。

\section{Netflix盈利模式分析}

Netflix的收入来源较为单一,其收入的98%来自于提供视频点播服务收取的订阅会员费,其余收入则来自美国国内单一费率邮寄DVD租赁业务的收入。由此我们看出,Netflix是一家典型的SaaS(软件即服务)
模式的公司。SaaS模式是一种通过网络提供软件服务的公司,即用户不再购买软件,而是通过公司提供的网络服务来享受软件的功能,用户无需学习如何使用、维护软件。具体到Netflix的例子则是,由Netflix
提供数量庞大的影视库,用户无需购买或租赁实体的DVD,并整理由实体DVD组成的个人影视库。用户只需缴纳订阅费,便可得到访问并观看Netflix影视库的权限,免去了购买、整理DVD的时间成本。

SaaS模式与传统模式存在差异,主要有如下几点区别:
\begin{itemize}
    \item SaaS模式需要用户以一定的时间周期支付订阅费用,通常这个周期是一个月/一个季度/一年,供应商一般会在第一次付费之前提供天数不等的试用;传统模式则是一次性买断。
    \item SaaS模式不需要用户对软件服务进行安装或配置,直接通过网络即可访问服务;传统模式则需要用户进行一定的安装或配置(例如配置蓝光播放器等)。
    \item SaaS模式会自动更新服务(影片上新不需要购买),而传统模式需要自行更新(影片上新需要购买)。
    \item 相比传统模式,SaaS模式的数据安全性更好(影片不会丢失)。
    \item SaaS模式中,提供商承担了维护费用,对用户更加友好;传统模式则需要自行维护,较为麻烦(DVD碟片需要妥善放置)
    \item SaaS模式提供的服务可以一次从电脑、电视、手机等多端享受,传统模式则只能一次通过一处享受。
\end{itemize}

Netflix自身的关键业务——影片租赁——是非常适合SaaS模式的。普通消费者观看电影的途径有限,在Netflix转型之前只有两种:影院观看、购买或租赁DVD。影院观看票价较贵,只能在上映期间观看,不够便捷;
购买或租赁DVD也有时间成本,缺乏快速接触到大量影视库的途径。流媒体模式则克服了上述两种观影方式的缺点,有着快速、大量、便宜的优势,消费者按需订阅即可,在不观看影片的时间段可以取消订阅,不需要
为了观影花费观影之外的时间(前往电影院的通勤时间、挑选/邮寄DVD的时间)。据此分析我们可以总结出SaaS模式相较于传统模式的优点:
\begin{itemize}
    \item 低成本。用户只需花费订阅费便可以在订阅期限内观看Netflix庞大影视库内的全部电影。
    \item 易于使用。登录Netflix便可以开始观影,不需要用户学习如何配置播放器、解码器、渲染器等,减少了学习成本,节省了用户的时间。
    \item 提供集成。Netflix与Roku、Nvidia等公司的电视盒子有合作,软件集成在硬件内。
    \item 不需要担心升级。Netflix会负责影片上新、技术升级的工作。
    \item 定价灵活。Netflix有三档会员,定价不同,对视频质量、可同时观看的屏幕分别做了限制,用户可按需选择适合自己的订阅模式。
\end{itemize}

当然,SaaS模式也有其劣势,例如数据安全风险(账号有被窃取的可能)、有限的集成功能(集成Netflix的硬件需要通过Netflix的认证)等,但总体而言,SaaS模式的优势远超其局限性,且局限性可以通过技术
手段的更新来克服。

2019年Netflix的财报显示,公司全年总收入201.56 亿美元,其中会员费收入为198.59 亿美元,这里有92.43 亿美元来自美国地区,有106.16 亿美元来自除美国外其他地区。Netflix用户的付费意愿远超其余流媒体服务商。
究其原因,是Netflix的核心资源足够吸引人。Netflix的影片库数量庞大,其中更是有独家制作的口碑影视,在Netflix独播,这些影视资源使得Netflix的用户付费意愿持续上升。Netflix也在加大对第一方制作影视的投入,
既可以完全掌握版权,也可以巩固自身的用户群。但同时我们也要注意到,第一方制作内容的成本是极高的,这也意味着Netflix必须维持现有的高增长才能保持未来持续的内容输出,巩固自身的用户群。Netflix的内容成本
占比从2013起一直保持在50%左右,加上客服、流量、支付平台手续费、营销以及研发等成本,Netflix虽然在2019年年收入实现了盈利,但长期来看Netflix仍然需要举债才能覆盖自身的业务成本,即Netflix实现了正现金流,
利润仍然是负数。面对透支未来收入用来制作内容的局面,如何保持自身的增长,使得未来的收入可以覆盖当年的摊销,是Netflix盈利模式需要关注的重点。

\section{Netflix商业模式创新}

管理哲学之父Charles Handy提出的“第二曲线理论”阐述道:任何一条增长的曲线都呈现抛物线的形状,企业要想持续发展,必须在到达抛物线的顶点之前创造一条新的曲线。这意味着一个企业想要持续成功,
不只需要创业之初的创新,而是需要不断地创新,如果只维持现有的成功路径,企业必然会走向衰落。典型的反面案例便是诺基亚,坚持功能机的它在智能机时代很快被淘汰。Netflix作为一家历时23年的企业,
三次突破了“二次曲线”的顶点,保持了持续的增长,堪称商业模式创新的典范。本节我们将分析Netflix的创新之处。

\subsection{第一次突破:租赁业务+早期O2O模式}
O2O模式,即Online To Offline,意为线下到线上,是一种将线下交易与线上平台相结合的商业模式。Netflix是最早实践这一模式的企业之一。在Netflix主营业务仍是租赁DVD的时代,其他厂商都是
由顾客前往实体店挑选并租赁DVD,在租期内看完再还到实体店。这样的模式对消费者很不友好,为了观看一部电影,消费者除花费租金外,还需要花费前往实体店的时间,如果超期,支付的罚金甚至比
DVD本身的价钱还要高。当时全美最大的DVD租赁公司之一的百事得,其收入的20%是客户缴纳的罚金。这些都引起了客户的不满。另一方面,实体店的DVD种类有限,由于仓储成本与运输成本较高,实体店
最多只能存放几百种DVD,用户的选择很少,对于一些冷门题材的爱好者非常不友好。

Netflix的第一次创新便是网上订阅与线下邮寄相结合的模式,克服了实体店租赁碟片的局限性:更多的选择、更少的时间成本、更便宜的平均资费、没有罚金。这样线上与线下相结合的模式,与近年来
在国内大力推广的“互联网+”极为相似,Netflix的创新是极具前瞻性的。Netflix此举颠覆了传统的收费模式,将客户的时间成本转化为公司的物流成本,再将物流成本分摊到每个客户的月费中,既实现了
便利用户的价值主张,又在成本可控的情况下达成了用户数的可观增长,是企业与用户的双赢。

\subsection{第二次突破:流媒体与推荐算法}
早在DVD邮寄租赁的阶段,Netflix便在自己的网站上开始尝试用算法向用户推荐影片,以帮助用户在Netflix庞大的影片库内定位到自己喜爱的影片。随着Netflix进军流媒体行业,推荐算法的准确性变得
比以往更重要。租赁DVD时代,用户如果租赁到不喜欢的电影需要等待寄回调换的时间,于是用户往往会花费更多的时间进行挑选。到了流媒体时代,用户对一部电影不感兴趣只需要关闭并打开另一部即可。
精准的推荐算法可以极大地改善用户体验,减少用户观影之外为了观影而耗费的时间,进而提高用户的留存率,使用户的付费意愿上涨。

Netflix的推荐算法名为CineMatch,这不是一种单一的算法,而是多种算法的结合。不同于早年间电视台对节目评估收视率的方式,CineMatch则把重心放在了用户上,评估被用户观看的电影。流媒体使得
Netflix掌握了庞大的用户数据:用户们在看什么,每一部电影的观影时长是多少,用户一天/一周花费在观影上的时间有多少,用户对每部电影的评分。大数据让Netflix为用户提供前所未有的精准个性化服务
成为了可能,但客观上也大大增加了CineMatch技术上实现的难度。为了尽可能优化CineMatch算法,Netflix在2006年至2009年举办了Netflix Prize竞赛,对能将CineMatch算法准确率提升10%的团队
提供百万美金的奖金。大赛的评测标准为预测指标与真实指标之间的均方根误差(越低越好),最终大赛的参与者们将上百种模型结合在一起,将预测指标从0.9525降低至0.8572。但大赛中提供的数据量远
小于真实的用户数据量,实际应用效果并不那么显著,最终Netflix只将其中最有效的两种算法:奇异值分解(SVD)和局限型玻尔兹曼机(RBM)应用到了CineMatch算法中。

下面对用户可感知的Netflix推荐系统做简单介绍:
\begin{enumerate}
    \item 个人影片推荐(PVR)\par 
    \setlength{\parindent}{2em}所有的Netflix推荐都基于该核心部分,它会分析你的偏好,在一页大约40行的推荐中掺杂个性化与非个性化的推荐(使用户发现新的偏好),每个用户的页面推荐均不一样。
    \item 现正热播(Trending Now)\par 
    \setlength{\parindent}{2em}Netflix会分析近期用户倾向于观看的影片类型,与个性化页面结合,生成推荐。(如情人节推荐爱情片)
    \item 榜单(Top-N Video Ranker)\par
    \setlength{\parindent}{2em}推荐系统会根据不同地区用户的偏好生成不同的地区前几位影片。
    \item 类似影片(Video-Video Smilarity)\par 
    \setlength{\parindent}{2em}算法会根据用户喜爱的片单推荐类似的影片。
    \item 继续观看(Continue Watching)\par
    \setlength{\parindent}{2em}Netflix会根据用户近几日的观看情况(观看时长、观看设备等)来分析当下用户最可能继续观看的影片。
\end{enumerate}

Netflix的算法也成为了Netflix的核心资产之一,是Netflix领军OTT行业必不可少的保证。

\subsection{第三次突破:版权与自制内容}
Netflix在初期只是内容的渠道商,只购买版权、不制作内容。然而,随着流媒体的崛起,厂商们都意识到了Netflix盈利的能力,版权费用日益攀升,购买并续费版权的成本长远来看愈来愈不划算。
在Starz取消了与Netflix的许可协议之后,Netflix意识到自有版权的重要性,只做渠道商会受制于人,于是Netflix决定自制内容。流媒体平台自制内容是又一次全新的尝试,业界没有先例。2013年,
Netflix推出的第一部自制剧《纸牌屋》大获成功,新增会员超过300万,收入较往年同期增长18%。Netflix又创新性地实施了一次性放出全集的放映方式,这样的方式不仅给观众观影体验更流畅,也给了
创作团队更大的创作自由。《纸牌屋》的成功奠定了Netflix主打自制剧的基调,在此之后Netflix每年都会制作大量自制内容,其中不乏精品。庞大的独占自制精品阵容使得Netflix用户留存意愿极高,每年大量的自制新剧也促进了用户的增长。
2017年,Netflix用户数量超过了美国有线电视用户数量,Netflix实际上成为了全球最大的媒体供应商。

Netflix目前的媒体库主要有三种内容:完全自制的内容,Netflix独家的第三方制作内容,传统媒体结束放映后上架的第三方制作内容。目前,Netflix正在扩充第一种内容,以减少后两种内容版权到期
对Netflix造成的冲击。由下图可以看出,自制内容增长迅速,其余内容增长放缓,自制内容占媒体库比例逐年降低。
\begin{figure}[!h]
    \centering
    \includegraphics[width=.6\textwidth]{content}
    \caption{自制内容与其他内容的比例分配}
\end{figure}

\subsection{Netflix其他创新:海报设计}
观众是否会观影有很大一部分是取决于影片的海报设计是否足够美观,能否直观传递影片信息。影院的海报可能并不适合流媒体,于是Netflix对海报进行重新设计,
对原始海报与改版海报进行A/B Test,观察哪张海报的点击率更高,更能吸引用户观看。Netflix随后将此技术自动化,设计了一套系统可以根据算法搭配电影中的物料
来自动生成不同的海报(不同的标题、不同的背景、不同的长宽比等),比较海报的效果,最终使用效果最好的一张海报。
\begin{figure}[!h]
    \centering
    \includegraphics[width=.6\textwidth]{abtest}
    \caption{Netflix A/B Test 流程图}
\end{figure}

Netflix无处不在的大数据算法应用在用户看不见的地方极大地提升了用户体验,稳定了用户群,这些算法以及积累出的经验成为了Netflix的核心资产之一,与独占内容
共同构成了Netflix的业务护城河。

\section{Netflix全球布局现状}

随着Netflix的业务不断扩展,美国地区的收入已不足以覆盖Netflix的成本,美国地区的付费用户增长也日渐缓慢,Netflix走向全球是公司发展的必然。
\begin{figure}[!h]
    \centering
    \includegraphics[width=.6\textwidth]{global}
    \caption{Netflix全球市场分布}
\end{figure}

对于向海外发展,Netflix的策略极为谨慎。它优先选择了文化差距小、地理位置近的加拿大试水。2010年,Netflix进入加拿大,从这里开始,它在5年间逐步
将业务拓展拓展到50个国家。Netflix将5年间的经验进行总结,深入分析了本地化的策略,在这之后快速扩张到190个国家和地区。Netflix的三阶段战略获得了成功,
在海外扩张中,本土化工作也完成地十分优秀。Netflix不仅为进驻的地区设计了本土化的界面与当地语言的配音和字幕,更为每个入驻的地区开发了专门的聚集内容。
以东亚市场为例,Netflix有日语剧集《全裸导演》《攻壳机动队》,韩语剧集《王国》《狩猎的时间》,华语剧集《谁是被害者》《周游记》。这些本土化的剧集对于打开市场有着极大的帮助。
此外,对于网络条件较差的地区,如印度,Netflix专门推出了下载功能。

针对不同地区的经济条件,Netflix也执行了分区定价,对于发展中国家,Netflix的月费相对发达国家较低。低廉的价格也是吸引新用户的重要手段,全球第一个推出半价的国家(马来西亚)和全球月费最低的国家(土耳其)
都位于亚洲地区。

\section{其他OTT服务商}

\subsection{强有力的竞争厂商}
Netflix作为渠道商制作内容获得的成功,不仅其他渠道商开始效仿,更有一些内容运营商开始做流媒体平台。最先跟进的是亚马逊的prime video,它在流媒体平台的动作与Netflix几乎同步,prime video
也拥有例如《致命女人》《黑袍纠察队》等爆款自制剧集,是Netflix有力的竞争对手之一。传统媒体也不愿坐以待毙,HBO、CBS都推出了各自的流媒体业务HBO Max和CBS All Access,HBO依靠《权力的游戏》
也拥有了稳定的用户群。有着电子租赁经验(iTunes)的Apple也推出了Apple TV+,不过由于平台的封闭性用户数量无法与Netflix相竞争。对Netflix最具威胁的是老牌内容制作商Disney+,手握大量版权的它不仅
打造出了爆款《曼达洛人》,还收回了与Netflix共同制作的漫威剧集版权,影视内容非常丰富。

\subsection{国内类似厂商}
国内被称为“中国Netflix”的厂商是爱奇艺,同属OTT服务厂商的还有腾讯视频和优酷视频。虽然主体业务类似,但国内厂商的商业模式与Netflix区别很大。国内的公司早期为了抢夺用户使用免费策略,导致
国内用户付费意愿偏低,厂商为了盈利加入了大量的广告,并推出“会员优先看”和“超前点播”的服务,这与Netflix只能付费使用、无广告、一次性放出全集的模式截然不同。这一点从收入构成也能看出,
Netflix几乎全部收入来自会员费,而爱奇艺有一大部分收入来自广告服务。

\section{结论:展望未来——Netflix的机遇与挑战}

毫无疑问,大屏娱乐的未来是属于OTT行业的。Netflix作为OTT行业的领军者,拥有业内最先进的推荐算法和庞大的自制内容媒体库,其核心资产是维持其地位的护城河。但我们同样也要注意到,Netflix至今
仍然没有实现整体盈利,在内容制作成本上的投入仍然居高。Netflix的股价也显示出了投资者的忧虑,其股价是最高位时的20%。Netflix的推荐算法也面临着如何使新用户快速感知等问题。不过在可预见的未来中,OTT行业的可增长空间很大,全球范围内有线电视依旧
是许多家庭娱乐方式的主流,潜在的用户数量不可小觑。Netflix用户的持续观看时间也在增加。随着科技的进步,越来越多的人会享受到科技的福利,Netflix所主导的高画质标准所需的硬件与网络条件也会有越来越多的人可以接触到,这也会提升Netflix高级别会员的数量,
增加Netflix的收入,拥有雄厚技术实力,不吝于研发经费的Netflix技术迭代的速度很可能远超想象,未来对Netflix是充满机遇的。

对于同行业的竞争者,与Netflix相似的模式意味着他们超越Netflix的可能性较低,巨大的成本也使得价格战是不现实的,一部分竞争者将会重归内容供应商,最终市场形成以Netflix主导的“一超多强”局面。
不论如何,未来是属于OTT服务的,其先进的商业模式与强大的技术力注定了传统电视业将被它取代。

% 在参考文献部分显示未引用的文献

\nocite{ref1, ref2}

% 生成参考文献
\bibliography{wpref}

\end{document}