\documentclass[lang=cn]{elegantpaper}

% 标题信息
\title{企业并购案例分析\\ \begin{large} 阿里巴巴网络技术有限公司收购网易考拉 \end{large} }
\author{赵之航 2018311178}
\date{\zhtoday}

\begin{document}

\maketitle

% \begin{abstract}
% 在这里写摘要。
% \keywords{关键词1,关键词2}
% \end{abstract}

\section{引言}
2019年9月6日,网易与阿里巴巴正式宣布双方达成战略合作,阿里巴巴集团以18.25亿美元全资收购网易旗下跨境电商平台考拉。
收购完成后网易考拉并入天猫国际进出口事业部,网易考拉品牌继续保留。本次交易完成后,收购方、被收购方以及整个市场格局都发生了相应的变化。

\section{并购双方简介}
\subsection{阿里巴巴}
阿里巴巴作为电商行业的领头羊,主要经营多元化的互联网业务, 包括促进B2B国际和中国国内贸易的网上交易市场 、网上零售和支付平台、网上购物搜索引擎,
以及以数据为中心的云计算服务,在新零售以及生态体系的搭建上具备一定的优势。
\subsection{网易考拉}
网易考拉海购原是网易旗下以跨境业务为主的综合型电商,通过产地批量直采和海外直邮两种方式为用户提供低价保真的海外商品。
网易首次公布电商的业绩是在2017年。被寄予厚望的网易电商,在2017年约占净收入总额的22%,近年来,通过网易考拉及网易严选的双线布局,
电商业务逐渐成为网易的核心业务之一。而网易最新公布的Q2财报显示,其电商收入增速已经从去年三季度的67.2%降至20%。
此外,网易考拉多次陷入“假货”风波,2017年底,网易考拉被中消协通报在2017年“双11”“海淘”商品假货名单中,所涉商品为自营的雅诗兰黛“小棕瓶”。
去年12月,有消费者投诉网易考拉所售加拿大鹅羽绒服为假货,尽管网易考拉在此后多次针对假货问题澄清,但是其对于供应链把控的薄弱,正是短板所在。
网易考拉业绩不断下滑,持续亏损,成了网易的“拖油瓶”业务,因此网易一直希望出售这样的的业务板块,及时止损,将钱花在更有前景的业务上。

业务层面上,网易的主营业务是游戏,网易考拉始终找不到与新零售完美融合的方式,考拉线下店、社交分销等相关尝试等均不顺利,
无论是线上和线下的融合,还是新技术在电商平台的应用,网易考拉始终找不到借力的方式和手段。

\section{并购背景及其动因}
\subsection{阿里巴巴收购动因}
\subsubsection{对海淘市场份额的抢夺}
阿里运营着天猫全球的跨境电子商务平台,通过天猫平台,阿里巴巴为国际公司在中国销售产品提供了一个门户。
选择收购网易考拉很大程度上是为了进一步巩固自己的地位,根据市场调查表明,网易考拉虽然近几年利润持续走低,
但仍然占据27.7%的市场,连续居于跨境电商市场份额的第一名。收购了网易海淘,一方面可以可以进一步巩固海淘市场,拥有最大的跨境电商用户群,
提供最全方位的跨境电商服务;另一方面也可以弥补自己在天猫国际上的不足。
\subsubsection{可以提高品牌供应链议价能力}
直营是网易考拉一直以来的核心优势,考拉在韩国、日本、欧洲、美国等地都有采购点,并花重金自建大量保税仓库。
数据显示,网易考拉在原有15个跨境综合试验区和试点城市中的绝大多数地方布局了仓储网络,保税仓面积超过100万平方米。
由于供应链建设需要投入大量时间精力,因此对天猫国际来说,收购网易考拉是“花钱买时间”的最佳选择。
收购后阿里不仅可以继承考拉目前的供应链和国际品牌资源,同时将掌握国内主要的海外用户群,有望依托其行业地位,大幅提升海外品牌与供应链议价能力。
\subsubsection{防御竞争对手}
在这场收购案定音落下之前,拼多多的身影也出现在了这场收购案里,,虽然它所扮演的角色更多的只是露了一下脸,但其中的深意或许可以猜测一下。
拼多多近几年发展迅猛,凭借着低廉的价格迅速占领了部分市场,此外由于微信允许拼多多在微信内传播,拼多多的获客成本极低。
除却拼多多,京东海囤全球当前也处于标品跨境进口B2C第一梯队,市占约12%紧跟天猫国际与考拉。
因此,考拉与天猫国际的合并,两个平台品牌供应链与用户规模相互强化,有望对京东以及拼多多等竞争对手未来进一步切入跨境进口B2C市场形成有效防御。
\subsection{网易考拉收购背景}
网易考拉业绩近两年不断下滑,持续亏损背后拖累的是网易整体的业绩表现。据网易财报,从2016年开始,电商业务日渐成熟,在网易总营收中的占比达到11.9%。
到2018年,占比已经高达28.64%。但网易在此期间的利润却是下滑的。2017年,网易净利润首次出现负增长,同比下滑8%;2018年净利润骤减40.3%,
为64.77亿元,甚至低于2015年。增收不增利或是由于网易电商板块的拖累,网易电商的毛利率只在个位数,而网易整体的毛利率却达到了60%以上。
可2018年网易的电商业务增长也陷入瓶颈。电商业务的年度增速从2017年Q4的175%骤降至2018年Q4的43%,进而拖累总营收增速滑落至2014年以来的最低水平。
到2019年Q2,网易电商业务营收为52.47亿元,同比增速已经降至20.2%,这也是网易最后一次披露电商业务成绩单。
因此出售这样的业务板块及时止损,把资金投入发展良好更有前景的业务上,是网易目前最优的选择。

由于前期供应链建设成本、为获取流量投入的营销成本巨大,网易考拉在快速增长中获利相当稀薄。
网易2018年第四季度财务数据显示,包括考拉在内的网易电商,虽然收入高达66.79亿元,但毛利润却不到3亿元,利润率仅为4.5%。
对核心业务并非电商的网易来说,这实在是一个不小的压力。而网易考拉被收购后,将获得阿里的资金、流量与商业支持,打造一流的跨境电商品牌,
继续为客户提供优质的跨境电商服务;网易自身则会将资源集中在优势领域,聚焦游戏等主营业务。

\section{并购概况}
\subsection{并购时间线}
2019年9月6日一早,阿里巴巴集团董事局主席兼首席执行官张勇现身网易杭州总部,
与网易集团董事局主席兼首席执行官丁磊一道宣布了阿里20亿美元收购网易考拉的消息。
在当天下午的管理层会议上,阿里表示,考拉不会裁员,六个月期间按照网易的方式运行。
下午,被收购的网易考拉PC端已经正式更名为“考拉海购”,手机端APP端也更名为“考拉海购”。

2019年9月27日,双方完成工商变更登记。
\subsection{交易模式}
网易考拉经营主体为杭州优卖网络科技有限公司,收购后股东为杭州阿里巴巴创业投资管理有限公司。阿里巴巴从原股东丁磊、朱静波处100%受让杭州优卖网络公司股权,
使得杭州优卖网络公司成为杭州阿里巴巴创业公司的全资子公司,实现阿里对考拉的控制。本次收购属于股权收购。
\subsection{并购过程}
\begin{figure}[!h]
    \centering
    \includegraphics[width=.6\textwidth]{1}
    \caption{并购过程}
\end{figure}

\section{并购结果}
\subsection{网易考拉被收购后得到的处理}
\begin{enumerate}
    \item 运营模式调整
    阿里巴巴集团以20亿美元全资收购网易旗下的跨境电商平台网易考拉,收购后考拉品牌将保持独立运营并与阿里旗下的天猫国际并行。
    \item 办公空间更改
    考拉从位于秋溢路的网易杭州园区搬到了对面——阿里滨江园区,考拉上千人整体迁移,并在2019年10月24日前完成住所地的工商变更,从办公地点上实现“空间整合”。
    \item{阿里巴巴指派管理团队}
    天猫进出口事业群总经理刘鹏兼任考拉CEO;天猫国际资深总监刘一曼现任考拉海购COO;原天猫品牌营销中心总监段玲出任考拉海购运营中心副总裁;
    阿里无线战略部署工程师蔡勇出任考拉海购产品中心负责人。这些人都曾在天猫、淘宝、聚划算等核心电商战场做出过突出贡献,
    另外阿里的胡瑜玲担任考拉收购案HRG,更对收购后的团队整合、人才体系建立以及文化理念形成都起到了关键性的作用。
    \item{原核心管理人选择性保留}
    尽管考拉管理层大多数离职或者转岗,但是最熟悉考拉核心业务的三位高管仍继续负责着相关的工作。
    其中,原CEO张蕾担任阿里巴巴集团CEO张勇的特别助理,帮助负责阿里非常重要的业务板块;原CTO朱静波代领考拉和阿里的相关技术团队;
    物流负责人刘煜也负责了与考拉有诸多交集的菜鸟自动化业务。核心团队的保留提高了效率,保留了优势,极有利于收购后的业务恢复与拓展。
    \item{原中层及基层员工整合进入阿里体系}
    在管理层队伍组建完成后,将考拉的中层员工整合进入阿里体系,打通层级,通过几百场定级会,逐一定级。
    最后,针对基层的年轻人,将开设专门的培养计划,同时陆续和天猫国际之间进行轮岗锻炼。
\end{enumerate}
\subsection{并购效应}
\subsubsection{战略协同效应}
从业务模式上来看,考拉以自营业务为主、平台为辅;天猫国际是平台为主、自营为辅。双方独立运营,网易考拉的供应链和客服系统可以和阿里供应链和客服系统相互打通,菜鸟可以和网易考拉各地保税仓相互打通,蚂蚁金服能够提供更多供应链金融服务,阿里强大的数据和信息技术都能够为网易考拉提供有力支撑。考拉良好的自营供应链模式,可以与阿里的中后台达成战略协同效应,实现1+1>2的效果。
\subsubsection{优势互补效应}
网易考拉与天猫国际实现有效融合,形成了各方面的优势互补:网易考拉以母婴产品起家,天猫国际则主打美妆产品,融合扩大了业务范围,增加了用户粘性;天猫国际平台擅长一线品牌首发,考拉则侧重二三线品牌,融合扩大了市场范围;天猫国际是开放平台,而网易考拉采取的是保税电商+直采自营,不同的业务模式和客户资源可以实现共享,特别是自营业务对阿里正在进行的线下实体店的开展起到了促进作用。
\begin{figure}[!h]
    \centering
    \includegraphics[width=.6\textwidth]{2}
    \caption{2019年上半年中国跨境电商平台市场份额分布}
\end{figure}
\subsubsection{经营的规模效应}
根据2019年上半年网络数据对跨境电商市场份额占有量的统计显示,网易考拉以27.7%的市场份额位居榜首,阿里巴巴旗下天猫国际和京东旗下海囤全球份额分别为25.1%和13.3%,阿里收购考拉之后,跨境电商份额将突破50%,占据市场半壁江山,规模效应将凸显,对其他竞争对手形成明显的规模优势。
\subsubsection{价值低估效应}
网易考拉与天猫国际的用户仍有差别,黑卡用户的消费力可观,未来考拉的品牌潜力很大。
\subsubsection{股价上涨效应}
阿里作为巨头,市盈率较高,并购后带动网易股价上涨2.86%。
\subsubsection{广告宣传效应}
在收购过程中提高了考拉海购的知名度,带动了企业的市场营销活动,对业务的开展有着积极的影响。


\end{document}