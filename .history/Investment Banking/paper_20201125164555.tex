\documentclass[lang=cn]{elegantpaper}

% 标题信息
\title{企业并购案例分析\\ \begin{large} ——阿里巴巴网络技术有限公司收购网易考拉 \end{large} }
\author{汪雨 \,王灿 \,赵之航 \,王杰峰 \,许嘉敏}
\date{\zhtoday}

\begin{document}

\maketitle

% \begin{abstract}
% 在这里写摘要。
% \keywords{关键词1,关键词2}
% \end{abstract}

\section{引言}
2019年9月6日,网易与阿里巴巴正式宣布双方达成战略合作,阿里巴巴集团以18.25亿美元全资收购网易旗下跨境电商平台考拉。
收购完成后网易考拉并入天猫国际进出口事业部,网易考拉品牌继续保留。本次交易完成后,收购方、被收购方以及整个市场格局都发生了相应的变化。

\section{并购双方简介}
\subsection{阿里巴巴}
阿里巴巴作为电商行业的领头羊,主要经营多元化的互联网业务, 包括促进B2B国际和中国国内贸易的网上交易市场 、网上零售和支付平台、网上购物搜索引擎,
以及以数据为中心的云计算服务,在新零售以及生态体系的搭建上具备一定的优势。
\subsection{网易考拉}
网易考拉海购原是网易旗下以跨境业务为主的综合型电商,通过产地批量直采和海外直邮两种方式为用户提供低价保真的海外商品。
网易首次公布电商的业绩是在2017年。被寄予厚望的网易电商,在2017年约占净收入总额的22%,近年来,通过网易考拉及网易严选的双线布局,
电商业务逐渐成为网易的核心业务之一。而网易最新公布的Q2财报显示,其电商收入增速已经从去年三季度的67.2%降至20%。
此外,网易考拉多次陷入“假货”风波,2017年底,网易考拉被中消协通报在2017年“双11”“海淘”商品假货名单中,所涉商品为自营的雅诗兰黛“小棕瓶”。
去年12月,有消费者投诉网易考拉所售加拿大鹅羽绒服为假货,尽管网易考拉在此后多次针对假货问题澄清,但是其对于供应链把控的薄弱,正是短板所在。
网易考拉业绩不断下滑,持续亏损,成了网易的“拖油瓶”业务,因此网易一直希望出售这样的的业务板块,及时止损,将钱花在更有前景的业务上。

业务层面上,网易的主营业务是游戏,网易考拉始终找不到与新零售完美融合的方式,考拉线下店、社交分销等相关尝试等均不顺利,
无论是线上和线下的融合,还是新技术在电商平台的应用,网易考拉始终找不到借力的方式和手段。

\section{并购背景及其动因}
\subsection{阿里巴巴收购动因}
\subsubsection{对海淘市场份额的抢夺}
阿里运营着天猫全球的跨境电子商务平台,通过天猫平台,阿里巴巴为国际公司在中国销售产品提供了一个门户。
选择收购网易考拉很大程度上是为了进一步巩固自己的地位,根据市场调查表明,网易考拉虽然近几年利润持续走低,
但仍然占据27.7%的市场,连续居于跨境电商市场份额的第一名。收购了网易海淘,一方面可以可以进一步巩固海淘市场,拥有最大的跨境电商用户群,
提供最全方位的跨境电商服务;另一方面也可以弥补自己在天猫国际上的不足。
\subsubsection{可以提高品牌供应链议价能力}
直营是网易考拉一直以来的核心优势,考拉在韩国、日本、欧洲、美国等地都有采购点,并花重金自建大量保税仓库。
数据显示,网易考拉在原有15个跨境综合试验区和试点城市中的绝大多数地方布局了仓储网络,保税仓面积超过100万平方米。
由于供应链建设需要投入大量时间精力,因此对天猫国际来说,收购网易考拉是“花钱买时间”的最佳选择。
收购后阿里不仅可以继承考拉目前的供应链和国际品牌资源,同时将掌握国内主要的海外用户群,有望依托其行业地位,大幅提升海外品牌与供应链议价能力。
\subsubsection{防御竞争对手}
在这场收购案定音落下之前,拼多多的身影也出现在了这场收购案里,,虽然它所扮演的角色更多的只是露了一下脸,但其中的深意或许可以猜测一下。
拼多多近几年发展迅猛,凭借着低廉的价格迅速占领了部分市场,此外由于微信允许拼多多在微信内传播,拼多多的获客成本极低。
除却拼多多,京东海囤全球当前也处于标品跨境进口B2C第一梯队,市占约12%紧跟天猫国际与考拉。
因此,考拉与天猫国际的合并,两个平台品牌供应链与用户规模相互强化,有望对京东以及拼多多等竞争对手未来进一步切入跨境进口B2C市场形成有效防御。
\subsection{网易考拉收购背景}
网易考拉业绩近两年不断下滑,持续亏损背后拖累的是网易整体的业绩表现。据网易财报,从2016年开始,电商业务日渐成熟,在网易总营收中的占比达到11.9%。
到2018年,占比已经高达28.64%。但网易在此期间的利润却是下滑的。2017年,网易净利润首次出现负增长,同比下滑8%;2018年净利润骤减40.3%,
为64.77亿元,甚至低于2015年。增收不增利或是由于网易电商板块的拖累,网易电商的毛利率只在个位数,而网易整体的毛利率却达到了60%以上。
可2018年网易的电商业务增长也陷入瓶颈。电商业务的年度增速从2017年Q4的175%骤降至2018年Q4的43%,进而拖累总营收增速滑落至2014年以来的最低水平。
到2019年Q2,网易电商业务营收为52.47亿元,同比增速已经降至20.2%,这也是网易最后一次披露电商业务成绩单。
因此出售这样的业务板块及时止损,把资金投入发展良好更有前景的业务上,是网易目前最优的选择。

由于前期供应链建设成本、为获取流量投入的营销成本巨大,网易考拉在快速增长中获利相当稀薄。
网易2018年第四季度财务数据显示,包括考拉在内的网易电商,虽然收入高达66.79亿元,但毛利润却不到3亿元,利润率仅为4.5%。
对核心业务并非电商的网易来说,这实在是一个不小的压力。而网易考拉被收购后,将获得阿里的资金、流量与商业支持,打造一流的跨境电商品牌,
继续为客户提供优质的跨境电商服务;网易自身则会将资源集中在优势领域,聚焦游戏等主营业务。

\section{并购概况}
\subsection{并购时间线}
2019年9月6日一早,阿里巴巴集团董事局主席兼首席执行官张勇现身网易杭州总部,
与网易集团董事局主席兼首席执行官丁磊一道宣布了阿里20亿美元收购网易考拉的消息。
在当天下午的管理层会议上,阿里表示,考拉不会裁员,六个月期间按照网易的方式运行。
下午,被收购的网易考拉PC端已经正式更名为“考拉海购”,手机端APP端也更名为“考拉海购”。

2019年9月27日,双方完成工商变更登记。
\subsection{交易模式}
网易考拉经营主体为杭州优卖网络科技有限公司,收购后股东为杭州阿里巴巴创业投资管理有限公司。阿里巴巴从原股东丁磊、朱静波处100%受让杭州优卖网络公司股权,
使得杭州优卖网络公司成为杭州阿里巴巴创业公司的全资子公司,实现阿里对考拉的控制。

阿里巴巴并购网易考拉,从行业隶属关系看,双方同属于网络行业,且在跨境进口零售的电商市场上具有竞争关系,属于同行业之间的并购,为横向并购;
在并购动机上,双方都是在自愿、合作、公开的前期下进行的,事先经过谈判协商最终达成一致的意愿,属于善意收购;阿里以20亿美元全资收购网易旗下跨境电商平台考拉,属于现金收购。
\subsection{并购过程}
\begin{figure}[!h]
    \centering
    \includegraphics[width=.46\textwidth]{1}
    \caption{并购过程}
\end{figure}

\section{并购结果}
\subsection{网易考拉被收购后得到的处理}
\begin{enumerate}
    \item 运营模式调整\par
    \setlength{\parindent}{2em}阿里巴巴集团以20亿美元全资收购网易旗下的跨境电商平台网易考拉,收购后考拉品牌将保持独立运营并与阿里旗下的天猫国际并行。
    \item 办公空间更改\par
    \setlength{\parindent}{2em}考拉从位于秋溢路的网易杭州园区搬到了对面——阿里滨江园区,考拉上千人整体迁移,并在2019年10月24日前完成住所地的工商变更,从办公地点上实现“空间整合”。
    \item{阿里巴巴指派管理团队}\par
    \setlength{\parindent}{2em}天猫进出口事业群总经理刘鹏兼任考拉CEO;天猫国际资深总监刘一曼现任考拉海购COO;原天猫品牌营销中心总监段玲出任考拉海购运营中心副总裁;
    阿里无线战略部署工程师蔡勇出任考拉海购产品中心负责人。这些人都曾在天猫、淘宝、聚划算等核心电商战场做出过突出贡献,
    另外阿里的胡瑜玲担任考拉收购案HRG,更对收购后的团队整合、人才体系建立以及文化理念形成都起到了关键性的作用。
    \item{原核心管理人选择性保留}\par
    \setlength{\parindent}{2em}尽管考拉管理层大多数离职或者转岗,但是最熟悉考拉核心业务的三位高管仍继续负责着相关的工作。
    其中,原CEO张蕾担任阿里巴巴集团CEO张勇的特别助理,帮助负责阿里非常重要的业务板块;原CTO朱静波代领考拉和阿里的相关技术团队;
    物流负责人刘煜也负责了与考拉有诸多交集的菜鸟自动化业务。核心团队的保留提高了效率,保留了优势,极有利于收购后的业务恢复与拓展。
    \item{原中层及基层员工整合进入阿里体系}\par
    \setlength{\parindent}{2em}在管理层队伍组建完成后,将考拉的中层员工整合进入阿里体系,打通层级,通过几百场定级会,逐一定级。
    最后,针对基层的年轻人,将开设专门的培养计划,同时陆续和天猫国际之间进行轮岗锻炼。
\end{enumerate}
\subsection{并购效应}
\begin{enumerate}
    \item{战略协同效应}\par
    \setlength{\parindent}{2em}从业务模式上来看,考拉以自营业务为主、平台为辅;天猫国际是平台为主、自营为辅。双方独立运营,网易考拉的供应链和客服系统可以和阿里供应链和客服系统相互打通,菜鸟可以和网易考拉各地保税仓相互打通,蚂蚁金服能够提供更多供应链金融服务,阿里强大的数据和信息技术都能够为网易考拉提供有力支撑。考拉良好的自营供应链模式,可以与阿里的中后台达成战略协同效应,实现1+1>2的效果。
    \item{优势互补效应}\par
    \setlength{\parindent}{2em}网易考拉与天猫国际实现有效融合,形成了各方面的优势互补:网易考拉以母婴产品起家,天猫国际则主打美妆产品,融合扩大了业务范围,增加了用户粘性;天猫国际平台擅长一线品牌首发,考拉则侧重二三线品牌,融合扩大了市场范围;天猫国际是开放平台,而网易考拉采取的是保税电商+直采自营,不同的业务模式和客户资源可以实现共享,特别是自营业务对阿里正在进行的线下实体店的开展起到了促进作用。
    \begin{figure}[!h]
        \centering
        \includegraphics[width=.6\textwidth]{2}
        \caption{2019年上半年中国跨境电商平台市场份额分布}
    \end{figure}
    \item{经营的规模效应}\par
    \setlength{\parindent}{2em}根据2019年上半年网络数据对跨境电商市场份额占有量的统计显示,网易考拉以27.7%的市场份额位居榜首,阿里巴巴旗下天猫国际和京东旗下海囤全球份额分别为25.1%和13.3%,阿里收购考拉之后,跨境电商份额将突破50%,占据市场半壁江山,规模效应将凸显,对其他竞争对手形成明显的规模优势。
    \item{价值低估效应}\par
    \setlength{\parindent}{2em}网易考拉与天猫国际的用户仍有差别,黑卡用户的消费力可观,未来考拉的品牌潜力很大。
    \item{股价上涨效应}\par
    \setlength{\parindent}{2em}阿里作为巨头,市盈率较高,并购后带动网易股价上涨2.86%。
    \item{广告宣传效应}\par
    \setlength{\parindent}{2em}在收购过程中提高了考拉海购的知名度,带动了企业的市场营销活动,对业务的开展有着积极的影响。
\end{enumerate}
\subsection{并购利弊分析}
\begin{enumerate}
    \item 对于阿里来说,网易考拉的品牌价值、供应链优势、仓储物流资源能够完美得与阿里形成互补关系,通过对考拉的收购,让阿里在进口零售电商市场的最大竞争对手变成自己的“队友”,完成重要的卡位战;同时,直营业务作为网易考拉的核心竞争力所在,阿里的收购让其在2019年三大战略的其中一条——升级直营业务迈出重要的一步。
    \item 对于考拉来说,阿里的收购将会使考拉获得充足的现金流和资本加持,能够重新聚焦主营业务,合理配置资源,归回高增长高利润的状态,再加上阿里本身的供应链、大数据、物流等优势的存在,考拉无疑会发挥出自己更大的价值。
    \item 对于消费者来说,两大巨头的合并,可以实现资源的高度集中,供应链的把控能力更强,最终的结果就是成本控制更有优势,最终能够买到更多高性价比的商品,且购物体验将会更加的丰富和多元。
\end{enumerate}
\subsection{理论支持}
\begin{enumerate}
    \item 市场势力理论。\par
    \setlength{\parindent}{2em}市场势力理论认为,并购活动的主要动因是可以借并购达到减少竞争对手的目的,进而来增强对经营环境的控制力,以提高市场占有率,是企业获得某种形式的垄断或寡占利润,并增加长期的获利机会。据Analysys易观发布的2019第1季度《中国跨境进口零售电商市场季度监测报告》显示,该季度天猫国际排名第一,市场份额为32.3%;网易考拉排名第二,份额为24.8%。阿里对考拉海购的收购将有利于阿里巴巴在中国跨境电子商务市场的影响力大大扩大。
    \item 自由现金流理论。\par
    \setlength{\parindent}{2em}在并购过程汇总,阿里巴巴将斥资20亿美元收购网易跨境电子商务业务网易考拉海购,阿里的自由现金流足够充裕。此外,阿里还将以7亿美元领投网易云音乐,意义重大。
    \item 效率理论。\par
    \setlength{\parindent}{2em}网易考拉与阿里的天猫国际同属跨境电商领域,并购后两家资源的整合可产生协同效应,即并购后的总体效益大于并购前两家企业的效益之和。
    \item 非效率管理理论。\par
    \setlength{\parindent}{2em}网易考拉近年来亏损严重,同时被曝有假货等负面新闻,相比之下阿里的管理层更加有效率,事实上,原考拉的管理层多数已离职或转岗,包括供应链负责人刘荣广(转岗)、仓储物流负责人刘煜(将转岗)、标品负责人冯小枫(离职)、CMO刘晓彬(离职)、全球工厂店负责人胡然(离职)、商业智能李勇(离职)、产品总监\&会员和用户中心负责人张通博(离职)、前台产品部负责人柯捷(离职)、客服负责人张明福(离职)等。阿里也关闭了与网易考拉的HC。
    \item 经营协同效应理论。\par 
    \setlength{\parindent}{2em}尽管并购后网易考拉仍单独运营,但幕后的管理工作以及供应链等体系必然会整合进阿里中,管理成本总体下降,产生规模经济性。未来阿里将会以考拉主打自营和天猫国际主打POP来稳固在电商领域的份额。
    \item 税收效应理论。\par 
    \setlength{\parindent}{2em}考拉的亏损在并购后可以使阿里利用税法中亏损递延条约合理避税。
\end{enumerate}

\section{案例启示}
\begin{enumerate}
    \item 并购是企业生产经营过程中增强竞争力的有效手段。现如今互联网工具、电商平台一直保持着良好的发展态势,且仍有巨大的发展潜力。要在这样强有力的竞争市场上站稳脚跟,就必须打造独一无二不可复制的优势,形成自己的核心竞争力。阿里巴巴在并购后,一方面集中人才、资本,增强了自身实力,提高了在行业产出中所占比重,另一方面,天猫服务体系的完善给用户带来了全面赋能的体验,在市场上极具吸引力。如果没有这次并购,天猫和考拉将以互相竞争的关系出现在消费者的选择里,优势难免被消耗,资源也会被浪费。
    \item 并购需要明确自己的目标。强势企业通过并购弱势企业实现产业格局的夸大以及市场份额的增加,弱势企业则投靠强势企业以获取充足的现金流,来谋求生存发展的机会。在并购时,特别是强势企业,要避免依靠强大的现金流,迅速占据市场而忽略自己的战略目标以及盲目并购的现象。例如腾讯入股搜狗是发展搜索引擎业务的需要,阿里入股微博是由于社交模块的开展,而这次阿里对考拉的收购,是对自己最核心的电商业务的补充。只有明确目标,才有机会获得成功。全球化是阿里的长期重要战略,阿里一直致力于通过数字化的平台为中国消费者提供全世界最好的商品和服务,对于进口商品的追求是迎合市场的必然结果,收购考拉正是在追求进口商品服务补充电商业务的目标下迈出的重要一步。
    \item 并购需要考虑各方面的风险。企业并购可能失败,从而遭受损失;另外,企业在成功实施并购行为后,未来收益仍具有不确定性。这些都是收购企业需要承担的风险。因此,收购方应当对自身的经济实力、所在行业的发展前景、发展战略等做出正确的价值判断,在选择目标公司是应尽量考虑其资产是否能与本公司的主营业务相配合。阿里巴巴资产规模大,经营业绩良好,旗下天猫国际直供海外原装进口商品,与网易考拉主打的跨境业务不谋而合,收购成功机会相对较大。但在收购后如何处理两大品牌构建的市场基础,如何在资源整合中不破坏生态体系的完整性,则是阿里在收购后需要为降低风险获取预期收益做出的努力。
    \item 并购需要做好相应的资源整合。并购企业与被并购企业在生产要素、服务要素、管理要素、文化要素上要实现高度一致,形成利益相关的命运共同体。企业的原先的战略、计划、组织结构、领导、、人力资源、财务、企业文化等都有必要进行系统性调整,以保证公司最佳的经营业绩。其中,阿里在对考拉的整合过程中注入了自己的管理层,又吸收了考拉的中坚力量,是实现血液互通、决定控制权增效的重要举措。而在文化理念的差异方面,实现二者的整合难度大,必须引起高度关注。
\end{enumerate}

\end{document}